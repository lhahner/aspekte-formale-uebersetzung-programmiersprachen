% Definieren Sie hier Ihre Abkürzungen anhand des DHGE Beispiels.
% Wenn Sie DHGE dann im Text verwenden, rufen sie einfach \ac{dhge} auf.
% LaTeX kümmert sich um den Rest.
% Für alles Weitere schauen Sie sich bitte die Dokumentation des Acro Packages an.

 \DeclareAcronym{ebnf}{
 	short = {EBNF},
 	long = {Erweiterte Backus-Naur-Form}
 }
 
  \DeclareAcronym{pom}{
 	short = {POM},
 	long = {Project Object Model}
 }
 
 \DeclareAcronym{mvc}{
 	short = {MVC},
 	long = {Model-View-Controller}
 }
 
  \DeclareAcronym{ide}{
 	short = {IDE},
 	long = {integrated development environment}
 }
 
   \DeclareAcronym{uml}{
 	short = {UML},
 	long = {Unified Modeling Language}
 }

	\DeclareAcronym{GCC}{
	short = {GCC},
	long = {GNU Compiler Collection}
	}
	
	\DeclareAcronym{jvm}{
	short = {JVM},
	long = {Java Virtual Machine}
	}
	
	\DeclareAcronym{ast}{
	short = {AST},
	long = {Abstract Syntax Tree}
	}
	
	\DeclareAcronym{JIT}{
		short = {JIT},
		long = { Just-In-Time Compiler}
	}
	
	\DeclareAcronym{zfs}{
		short = {ZFS},
		long = { zSeries Filesystem }
	}
 
 \DeclareAcronym{hfs}{
 	short = {HFS},
 	long = { Hieracical Filesystem }
 }
 
  \DeclareAcronym{pli}{
 	short = {PL/I},
 	long = { Programming Language One }
 }
 
   \DeclareAcronym{MIPS}{
 	short = {MIPS},
 	long = { Million Instructions per Second }
 }

  
  