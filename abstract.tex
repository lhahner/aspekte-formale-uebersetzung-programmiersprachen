Im Jahr 2023 besteht die Möglichkeit, die Spracheinschränkung einer Programmiersprache
aufzuheben und nur mithilfe natürlicher Sprache einem Computer
Steuerbefehle zu erteilen. Large Language Models (LLM) nehmen natürliche
Sprache als Eingabe und können formale Ausdrücke diverser Programmiersprachen
als Ausgabe erzeugen. Im Gegensatz zu herkömmlichen Compilern
und Transpilern sind LLMs jedoch auf natürliche Sprachen spezialisiert. Es wird
diskutiert, ob es möglich ist, mit einem LLM ein Programm nahezu vollständig
von einer Programmiersprache in eine andere zu übersetzen. Es wird untersucht,
inwiefern dies technisch unzuverlässig ist und ob bzw. wie sich Fehler
vermeiden lassen. Dabei wird auch der Unterschied zum technischen Vorgehen
eines Transpilers verdeutlicht. Es wird klar, dass die Grenzen der Übersetzung
natürlichsprachlicher Ausdrücke in formale darin bestehen, dass eine formale
Grammatik direkten Einfluss auf die Ausführungsbedingungen des Computers
in Bezug auf Speicherverwaltung, Prozessverwaltung und -steuerung hat.
Verdeutlicht wird das an dem Beispiel der Programmiersprache PL/I. Da
Pl/I ein typisches Beispiel für eine legacy Programmiersprache ist, die über
mehrere Generationen weitergegeben wurde. Dabei wird der PL/I Code in
Java Code transformiert. Ein Fokus liegt auf der Deklaration von Datentypen
in den jeweiligen Sprachen.