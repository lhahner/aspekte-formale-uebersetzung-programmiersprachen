Die vorangegangenen Arbeit 'Softwaregestützte Transformation von legacy Code in moderne Programmiersprachen' behandelt Übersetzungsmethoden der Datentypen von PL/I zu Java. Dabei wurde festgestellt, dass die Übersetzung von Datentypen mithilfe eines Transpilers auf unterschiedliche Art und Weise gestaltet werden kann. Die Arbeit bestand zum großteil aus dem Vergleich der Übersetzungsgestaltung eines Large Language Models (LLM) und eines selbst geschriebenen Transpilers. Der Transpiler der im Zuge der Arbeit entwickelt wurde soll mit dieser Arbeit erweitert werden.

Das bisherige Programm deckt lediglich die Übersetzung von Datentypen ab, jedoch nicht das Übersetzen von ganzen Prozeduren mit Verzweigungen und anderen Kontrollstrukturen. Aus diesem Grund wird der bestehende Transpiler so erweitert, dass eine Übersetzung anderer PL/I-Sprachkonstrukte in Java möglich ist. Dies umfasst auch die Umstrukturierung des aktuellen Programms, die Weiterentwicklung und Aufnahme neuer Ausdrücke der Sprache PL/I aus der Sprachreferenz von IBM, sowie die Diskussion über deren Repräsentation in Java. Dabei werden Softwarewerkzeuge wie JavaCC, Spring Boot, Maven und JUnit verwendet und im Zusammenhang mit der Erarbeitung beschrieben. 

Am Ende wird eine Benutzerschnittstelle dargestellt, die die Nutzung des Transpilers für PL/I-Entwickler oder Modernisierungsspezialisten erleichtert. Schlussendlich sollen Performance-Tests und daraus resultierende Optimierungsmöglichkeiten potentielle Mitentwickler anregen das bestehende Programm zu erweitern.

% In dem Buch "Modern Compiler Design" wurde beschrieben das die erzeugung eines Lexers mithilfe von automatisch generierten Lösungen wie JavaCC ineffzient sein kann. Dieser Punkt könnte sich im nachhinein noch als Nützlich erweisen, wenn am Ende über Optimierungspotential der Software diskutiert wird.