In der vorangegangenen Arbeit 'Softwaregestützte Transformation von legacy Code in moderne Programmiersprachen' wurde ein Transpiler entwickelt , der in dieser Bachelorarbeit erweitert wird.
Dies umfasst die Umstrukturierung des aktuellen Programms, die Weiterentwicklung und Aufnahme neuer Ausdrücke der Sprache PL/I aus der Sprachreferenz von IBM, sowie die Diskussion über deren Umwandlung  nach Java. Dabei werden die Softwarewerkzeuge JavaCC, Spring Boot, Maven und JUnit verwendet. Architekturentscheidungen orientieren sich an geeigneten Entwurfsmustern.
Außerdem wird eine Benutzerschnittstelle dargestellt, die die Nutzung des Transpilers erleichtert. Als qualitätssichernde Maßnahmen werden Fehlerklassen, Unit-, Integrations- und Performancetests entworfen und ausgeführt.

In the previous thesis 'Softwaregestützte Transformation von legacy Code in moderne Programmiersprachen', a transpiler was developed which is extended in this bachelor thesis.
This includes the restructuring of the current program, the further development and inclusion of new expressions of the PL/I language from the IBM language reference, as well as the discussion of their conversion to Java. The software tools JavaCC, Spring Boot, Maven and JUnit are used for this. Architectural decisions are based on suitable design patterns.
In addition, a user interface is presented that facilitates interaction with the transpiler. Errorclasses, unit-, integration- and performancetests are designed and executed as quality assurance measures.


% In dem Buch "Modern Compiler Design" wurde beschrieben das die erzeugung eines Lexers mithilfe von automatisch generierten Lösungen wie JavaCC ineffzient sein kann. Dieser Punkt könnte sich im nachhinein noch als Nützlich erweisen, wenn am Ende über Optimierungspotential der Software diskutiert wird.
