Wird 'Ich bin' ins Englische übersetzt, lautet die Übersetzung 'I am'. Wird nun 'Du bist', ins Englische übersetzt, wird daraus 'You are'. Hingegen der ersten übersetzung geht bei der zweiten etwas verloren. 

Mit dem Ausdruck 'Du bist', beschreibe ich im deutschen den Zustand oder das Verhalten eines anderen. Wenn ich im Englische sage 'You are', dann scheint es so als wenn die Person die ich anspreche mehrfach existiert, denn 'You are' ist Plural und übersetzt den deutschen Text nicht exakt. 

Um den deutschen Text exakt zu übersetzen, bleiben Englische Grammatikregeln unberücksichtigt. Ich müsste einen Ausdruck, wie etwa 'You am' formen, um das deutsche 'Du bist' exakt zu übersetzen. 

Eine Übersetzung von natürlichen Sprachen bei der semantische und syntaktische Korrektheit erfolgt, ist somit kaum exakt.

Gleiches gilt auch für Programmiersprachen. In meiner bisherigen Arbeit 'Softwaregestützte Transformation von legacy Code', habe ich mich näher mit Übersetzungsmethoden von Programmiersprachen auseinandergesetzt. In dieser Arbeit wurde deutlich das Bestandteile wie Datentypen der Programmiersprache PL/I, nicht nativ so übersetzt werden können das diese die gleiche semantische und syntaktische Bedeutung in Java haben. Es braucht ein Programm welches die Transformation von Bedeutung und verwendung von PL/I Code in Java übernimmt. Die erarbeitung dieses Transformationsprogramms, soll der Bestandteil dieser Arbeit sein.

--> PL/I Datentypen übersetzt in PA4, jetzt möchte ich das Modell erweitern. 
--> Mit Syntaktischen Produkten erweitern und übersetzen.
--> JavaCC wird zur erarbeitung des Parsers verwendet.

- PL/I Programm in Java Programmcode übersetzen
- Prozeduren zu überführen
- Eigendefinierte Grammatik anhand des PL/I Language Referenz
- Zwischencode 
