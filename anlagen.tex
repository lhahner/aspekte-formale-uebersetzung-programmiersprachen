

% Tests
\begin{figure}[H]
\begin{lstlisting}[language=Java, caption=mapArithmetic Test, label={lst:pliarithmeticexpression_test}]
@Test
@DisplayName("Identifier &  Decimal Mapping")
void mapChildNodes_checkIdentifierMapping() {
	DeclarationMapper declarationMapper = new DeclarationMapper();
	String identifier = "var_3";
	String javaExpression = "@Decimal(5) double";
	String decimalExpression = "test_1_package: PACKAGE;" + "	DCL var_3 FIXED DECIMAL(5)" + "END test_1_package;";
	
	try {
		
		InputStream stream = new ByteArrayInputStream(decimalExpression.getBytes(StandardCharsets.UTF_8));
		Pl1Parser pl1parser = new Pl1Parser(stream);
		SimpleNode program = pl1parser.program();
		SimpleNode varNode = (SimpleNode) program.jjtGetChild(0).jjtGetChild(1);
		
		declarationMapper.mapChildNodes(varNode);
		declarationMapper.mapArithmetic((SimpleNode)varNode.jjtGetChild(1).jjtGetChild(0));
		
	} catch (Exception e) {
		
		e.printStackTrace();
		
	}
	assertEquals(identifier, declarationMapper.getIdentifier());
	assertEquals(javaExpression, declarationMapper.getType());
}		
\end{lstlisting}
\end{figure}


\begin{figure}[H]
	\begin{lstlisting}[language=PL/I, caption=mapHeadNode Test, label={lst:procchar_test}]
		@Test
		@DisplayName("Map Head Node")
		void mapHeadNode_Identifier() {
			HeadMapper headm = new HeadMapper();
			
			String identifier = "A10_Revenue";
			String paraList = "(Object rev_1)";
			String type  = "@Char(5) double";
			
			String javaExpression = "public void proc_1()";
			String decimalExpression = "testproc_1_package: PACKAGE;" 
			+ "	A10_Revenue: PROC(rev_1) RETURNS(DECIMAL(5)) OPTIONS(INLINE);"
			+ "	END proc_1;" 
			+ "END testproc_1_package;";
			
			try {
				
				InputStream stream = new ByteArrayInputStream(decimalExpression.getBytes(StandardCharsets.UTF_8));
				Pl1Parser pl1parser = new Pl1Parser(stream);
				SimpleNode program = pl1parser.program();
				SimpleNode varNode = (SimpleNode) program.jjtGetChild(0).jjtGetChild(1);
				
				headm.mapHeadNode((SimpleNode)varNode.jjtGetChild(0));
				
			} catch (Exception e) {
				
				e.printStackTrace();
				
			}
			assertEquals(identifier, headm.getIdentifier());
			assertEquals(paraList, headm.getParameter());
			assertEquals(type, headm.getType());
		}		
	\end{lstlisting}
\end{figure}

\begin{figure}[H]
	\begin{lstlisting}[language=Java, caption=mapCallStatement Test, label={lst:callstatement_test}]
	@Test
	@DisplayName("Test Call Statement")
	void mapCallStatement_withoutParameter() {
		String identifier = "proc_2";
		String parameter = "(\"test_1\",1)";
		String charExpression = 
		"testcall1_package: PACKAGE;" 
		+ "proc_4: PROC;"
		+ "	CALL proc_2;" 
		+ "END proc_4;" 
		+ "END testcall1_package;";
		
		try {
			
			InputStream stream = new ByteArrayInputStream(charExpression.getBytes(StandardCharsets.UTF_8));
			Pl1Parser pl1parser = new Pl1Parser(stream);
			SimpleNode program = pl1parser.program();
			SimpleNode varNode = (SimpleNode) program.jjtGetChild(0).jjtGetChild(1).jjtGetChild(1).jjtGetChild(0);
			
			CallMapper cm = new CallMapper();
			cm.mapCallStatement(varNode);
			assertEquals(identifier, cm.getIdentifier());
			assertEquals(parameter, cm.getParameter());
			
		} catch (Exception e) {
			
			e.printStackTrace();
			
		}
	}
	\end{lstlisting}
\end{figure}

\begin{figure}[H]
	\begin{lstlisting}[language=Java, caption=setExpressionList Test, label={lst:branchmapping_test}]
	@Test
	@DisplayName("Tests Mapping of Boolean Expression")
	void setExpressionList_secondTest() {
		
		BooleanExpressionMapper bm = new BooleanExpressionMapper();
		String javaExpression = "(var_1 < var_2)";
		String booleanExpression = 
		"booleanTest1Package: PACKAGE;"
		+ "booleanTest1Proc: PROC;"
		+ "	IF (var_1 < var_2) THEN;"
		
		+ "	END;"
		+ "END booleanTest1Proc;"
		+ "END booleanTest1Package;";
		ArrayList<String> expressionList = new ArrayList<String>();
		try {
			
			InputStream stream = new ByteArrayInputStream(booleanExpression.getBytes(StandardCharsets.UTF_8));
			Pl1Parser pl1parser = new Pl1Parser(stream);
			SimpleNode program = pl1parser.program();
			SimpleNode varNode = (SimpleNode) program.jjtGetChild(0).jjtGetChild(1).jjtGetChild(1).jjtGetChild(0).jjtGetChild(0);
			
			bm.setExpressionList(varNode, expressionList);
			bm.mapBooleanExpression(expressionList);
			
			assertEquals(javaExpression, bm.getExpression());
			
		} catch (Exception e) {
			
			e.printStackTrace();
			
		}
	}
	\end{lstlisting}
\end{figure}

\begin{figure}[H]
	\begin{lstlisting}[language=Java, caption=mapUntilNode Test, label={lst:untilwhile_test}]
		@Test
		@DisplayName("Tests Mapping of Boolean Expression")
		void mapUntilNode_test() {
			
			BooleanExpressionMapper bm = new BooleanExpressionMapper();
			String javaExpression1 = "(var_1 < var_2)";
			String javaExpression2 = "(!var_1 = var_2)";
			String javaExpression3 = "while(" + javaExpression2 + ")";
			String untilLoop = 
			"looptestpack: PACKAGE;"
			+ "looptestproc: PROC;"
			+ "DO"
			+ "	WHILE(var_1 < var_2) UNTIL(var_1 = var_2);"
			+ "END;"
			+ "END booleanTestProc;"
			+ "END booleanTestPackage;";
			
			try {
				
				InputStream stream = new ByteArrayInputStream(untilLoop.getBytes(StandardCharsets.UTF_8));
				Pl1Parser pl1parser = new Pl1Parser(stream);
				SimpleNode program = pl1parser.program();
				
				SimpleNode whileBool = (SimpleNode) program.jjtGetChild(0).jjtGetChild(0).jjtGetChild(0).jjtGetChild(0);
				bm.translate(whileBool);
				assertEquals(javaExpression1, bm.getExpression());
				
				SimpleNode untilBool = (SimpleNode) program.jjtGetChild(0).jjtGetChild(0).jjtGetChild(0).jjtGetChild(1).jjtGetChild(0);
				bm.translate(untilBool);
				assertEquals(javaExpression2, bm.getExpression());
				
				
				assertEquals(javaExpression3, new EndMapper().getClosingExpression());
				
			} catch (Exception e) {
				
				e.printStackTrace();
				
			}
		}
	\end{lstlisting}
\end{figure}

\begin{figure}[H]
	\begin{lstlisting}[language=Java, caption=mapDisplayNode Test, label={lst:displayreply_test}]
		@Test
		@DisplayName("Read and write to stdio")
		void mapDisplayNode_testMessage() {
			DisplayMapper dm = new DisplayMapper();
			
			String message = "\"Hello World\"";
			String parameter = "message_para";
			String displayStatement = 
			"displayTest: PACKAGE;"
			+ "DCL message_para CHAR(5);"
			+ "displayProc: PROC;"
			+ " DISPLAY('Hello World') REPLY (message_para);"
			+ " END displayProc;"
			+ "END displayTest;";
			
			try {
				
				InputStream stream = new ByteArrayInputStream(displayStatement.getBytes(StandardCharsets.UTF_8));
				Pl1Parser pl1parser = new Pl1Parser(stream);
				SimpleNode program = pl1parser.program();
				SimpleNode varNode = (SimpleNode) program.jjtGetChild(0).jjtGetChild(2).jjtGetChild(1).jjtGetChild(0);
				
				dm.mapDisplayNode((SimpleNode) varNode);
				
			} catch (Exception e) {
				
				e.printStackTrace();
				
			}
			assertEquals(message, dm.getMessage());
			assertEquals(parameter, dm.getParameter());
			
		}
	\end{lstlisting}
\end{figure}

\begin{figure}[H]
	\begin{lstlisting}[language=Java, caption=mapWriteNode Test, label={lst:javawriter_test}]
		@Test
		void mapWriteNode_expectedJavaExpression_withIdToWriteTo() {
			try {
				String charSequence = "str_1";
				String[] values = { filename, charSequence };
				writeStatment.jjtSetValue(values);
				
				assertEquals("BufferedWriter writer = new BufferedWriter(new FileWriter(" + filename
				+ ")); \n writer.write(" + charSequence + ");", writeMapper.translate(writeStatment));
			} catch (Exception e) {
				e.printStackTrace();
			}
		}
	\end{lstlisting}
\end{figure}

% Benchmark Programm
\begin{lstlisting}[language=PL/I, caption=Probdivision-Algorhitmus PL/I, label={lst:pliprobedivision}]
PACKBA: PACKAGE;
	DCL Z DECIMAL(5);
	DCL MESSAGE CHAR(5);
	
	PROCBA: PROC(M) RETURNS(DECIMAL(5));
		IF(M<2) THEN;
			RETURN 1;
		END;
		IF(MOD(M, 2) = 0) THEN;
			IF(M=2) THEN;
				RETURN 0;
			ELSE
				RETURN 1;
			END;
		END;
		IF(MOD(M, 3) = 0) THEN;
			IF(M = 3) THEN;
				RETURN 0;
			ELSE
				RETURN 1;
			END;
		END;
		Z=5;
		DO
			WHILE(Z*Z<=M);
				IF(MOD(M, Z) = 0) THEN;
					RETURN 1;
				END;
		Z = Z + 2;
		IF(MOD(M,Z) = 0) THEN;
			RETURN 1;
		END;
	Z = Z + 4;
	END;
	END PROCBA;
END PACKBA;
\end{lstlisting}

\begin{lstlisting}[language=Java, caption=Probdivision-Algorhitmus Java, label={lst:javaprobedivision}]
public class TranspiledExamplePliProgram implements IProgram  { 
	public @Decimal(5) double Z;
	public @Char(5) String MESSAGE;
	
	@Decimal(5)
	public double PROCBA(Object M){
		if( M < 2 ){
			return 1;
		}
		if( M%2 == 0 ){
			if( M == 2 ){
				return 0;
			} else{ 
				return 1;
			}
		} 
		if(M%3 == 0) {
			 if( M == 3 ){
			 	return 0;
			 } else{ 
			 	return 1;
			 }
		}
		Z = 5;
		while(Z * Z <= M){ 
			if(M%Z == 0) {
				return 1;
			}
			Z = Z+2;
			if(M%Z == 0){
				return 1;
			}
			Z = Z+4;
		}
	}
}
\end{lstlisting}

\dhgefigure[h]{test-21.png}[scale=0.5]{Testergebnis PL/I-Programm: 2,147,483,647}{fig:test2147}[][]

\dhgefigure[h]{test-999.png}[scale=0.5]{Testergebnis PL/I-Programm: 999,999,000,001}{fig:test999}[][]

\dhgefigure[h]{test-672.png}[scale=0.5]{Testergebnis PL/I-Programm: 67,280,421,310,721}{fig:test672}[][]

\dhgefigure[h]{test-10000.png}[scale=0.5]{Testergebnis PL/I-Programm: 10,000,000,002,065,383}{fig:test1000}[][]



\begin{lstlisting}[language=Java, caption=Angepasste Java-Programm, label={lst:javaprobedivisionang}]
public class TranspiledExamplePliProgram implements IProgram {
	
	public BigDecimal Z;
	public String MESSAGE;
	
	public double PROCBA(BigDecimal M) {
		if (M.compareTo(BigDecimal.valueOf(2)) < 0) {
			return 1;
		}
		//M % 2 == 0
		if (M.remainder(BigDecimal.valueOf(2)) == BigDecimal.valueOf(0)) {
			if (M == BigDecimal.valueOf(2)) {
				return 0;
			} else {
				return 1;
			}
		}
		if (M.remainder(BigDecimal.valueOf(3)) == BigDecimal.valueOf(0)) {
			if (M == BigDecimal.valueOf(3)) {
				return 0;
			} else {
				return 1;
			}
		}
		Z = Z.valueOf(5);
		while (Z.multiply(Z).compareTo(M) <= 0 ) {
			System.out.println(Z.toString());
			if (M.remainder(Z) == BigDecimal.valueOf(0)) {
				return 1;
			}
			Z = Z.add(BigDecimal.valueOf(2));
			if (M.remainder(Z) == BigDecimal.valueOf(0)) {
				return 1;
			}
			Z = Z.add(BigDecimal.valueOf(4));
			
		}
		return M.doubleValue();
	}
}

\end{lstlisting}