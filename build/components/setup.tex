% FONT SETUP
% schönere Fonts, aber optional. Zum deaktivieren CFANCYFONTS in config.tex auf 0 setzen
\if\CFANCYFONTS 1
    \lstset{
        basicstyle=\ttfamily,
        breaklines=true
    }
\fi


% VARIABLE SETUP
% die veraltete CAUTHOR Variable wird automatisch befüllt
\def\CAUTHOR{\CAUTHORVOR\ \CAUTHORNACH}

\if\CARBEIT B
    \def\BETREUER{Gutachter}
\else
    \def\BETREUER{Betreuer}
\fi


% DOCUMENT SETUP
\onehalfspacing % 1.5 line spacing
% TODO: sollte theoretisch keine Verwendung mehr haben
% \widowpenalty10000
% \clubpenalty10000


% INHALTSVERZEICHNIS SETUP
\renewcommand{\contentsname}{Inhaltsverzeichnis}
\cftsetindents{section}{0em}{4em}
\cftsetindents{subsection}{0em}{4em}
\cftsetindents{subsubsection}{0em}{4em}
\setcounter{tocdepth}{3}
\setcounter{secnumdepth}{5}


% ABBILDUNGEN UND TABELLEN SETUP
\renewcommand{\listfigurename}{Abbildungsverzeichnis}
\renewcommand{\listtablename}{Tabellenverzeichnis}

\addto{\captionsngerman}{
    \renewcommand*{\figurename}{Abb.}
    \renewcommand*{\tablename}{Tab.}
}

\addtocontents{lof}{\linespread{2}\selectfont}
\addtocontents{lot}{\linespread{2}\selectfont}

\makeatletter
\renewcommand{\cftfigpresnum}{Abb. }
\renewcommand{\cfttabpresnum}{Tab. }

\setlength{\cftfignumwidth}{2cm}
\setlength{\cfttabnumwidth}{2cm}

\setlength{\cftfigindent}{0cm}
\setlength{\cfttabindent}{0cm}
\makeatother

% MODUS KUSCHE: ABBILDUNGEN UND TABELLEN SETUP
% im Kusche-Mode sollen Abbildungen nach Kapitel.lfd nummeriert werden
\if\CKUSCHE 1
    \counterwithin{figure}{section}
    \counterwithin{table}{section}
\fi


% CAPTION SETUP
\captionsetup{
    font = small,
    labelfont = bf,
    singlelinecheck = false,
    skip = 10pt,
    belowskip = 0pt
}


% CITATION SETUP
\renewcommand*{\labelalphaothers}{\textsuperscript{}}


% HEADERS & FOOTERS
\pagestyle		{fancyplain}
\fancyhf		{}
\renewcommand	{\headrulewidth}{0pt}
\renewcommand	{\footrulewidth}{0pt}
\setlength		{\headheight}{15pt}

% KUSCHE MODE: HEADERS & FOOTERS

\fancyfoot      [R]{\thepage} % nach den neuen Anforderungen sind Seitenzahlen unten rechts, geht d'accord mit dem Kusche-Mode
\if\CKUSCHE 1
    \fancyfoot      [L]{\leftmark} % im Kusche-Mode erscheint linksbündig das Kapitel in der Fußzeile
\fi


% FOOTNOTE SETUP
\renewcommand{\footnotelayout}{\hspace{0.5em}}

% COUNTER
% Zweck: in römischen Zahlen weiter zählen, nachdem der Counter von arabisch zurück geändert wird
\newcounter{savepage}


% SECTION SETUP
% sections sollen mit Seitenumbruch beginnen

\let\stdsection\section
\renewcommand\section{\newpage\stdsection}


% MATHRM ADJUSTMENTS
% Überschreiben von \mathrm{} -> einheitlichen Abstand einfügen
\let\oldMathrm\mathrm
\renewcommand{\mathrm}[1]{\,\oldMathrm{#1}}

% PATH SETUP
% root ist ist das neue Arbeitsverzeichnis
\makeatletter
\def\input@path{{../}{path1/}}
\makeatother

\graphicspath	{{../assets/img/}}


% AUTO REMOVE/INSERT ABBILDUNGSVERZEICHNIS & TABELLENVERZEICHNIS
% Conditionals um AbbildungVZ und TabellenVZ nur zu rendern, wenn sie nicht leer sind
\newtotcounter{figCount}
\newtotcounter{tabCount}
\let\oldTabTOC=\table
\let\oldFigTOC=\figure
\renewcommand{\figure}{\stepcounter{figCount}\oldFigTOC}
\renewcommand{\table}{\stepcounter{tabCount}\oldTabTOC}

\newcommand{\conditionalLoF}{
    \ifnum\totvalue{figCount}>0
        \addcontentsline{toc}{section}{\listfigurename}
        \listoffigures
        \cleardoublepage
    \fi
}
\newcommand{\conditionalLoT}{
    \ifnum\totvalue{tabCount}>0
        \addcontentsline{toc}{section}{\listtablename}
        \listoftables
        \cleardoublepage
    \fi
}


% ANLAGENVERZEICHNIS SETUP
% definiert eine neue Liste für das Anlagenverzeichnis
\newcommand{\listanlageverzeichnis}{\vspace*{-20pt}}
\newlistof{anlagen}{alt}{\listanlageverzeichnis}

% Befehl welcher ein Item dem Anlagenverzeichnis hinzufügt
\newcommand{\addItemToAnlageverzeichnis}[1]{%
    \def\fig{fig}
    \def\tab{tab}

    \ifx\fig\typeOfCap
        \def\type{\thefigure}
        \def\name{Abb.\hspace{8pt}}
    \else \ifx\tab\typeOfCap
            \def\type{\thetable}
            \def\name{Tab.\hspace{10pt}}
        \fi
    \fi
    \setcounter{anlagen}{\type}

    % only here because the \type-counter is one lower (later it will count up like normal)
    % -> reason... it's called too early but can't called later because of dependencies other types
    % works only in the last section of the paper so it should be fine :)
    \refstepcounter{anlagen}

    \addcontentsline{alt}{anlagen}
    {\name\protect\numberline{\theanlagen}\quad#1}\par
}

\newenvironment{longfigure}{\captionsetup{type=figure}}{}

% AUTO REMOVE/INSERT Anlagenverzeichnis
\newtotcounter{anlagenentries}  % stepCounter within table and figure to check if used
\newcommand{\renewFigTabCap} {
    % \caption ruft zusätzlich \addItemToAnlageverzeichnis auf
    \let\oldCap=\caption
    \renewcommand{\caption}[1]{\addItemToAnlageverzeichnis{##1}\oldCap{##1}}

    % redefine table and figure -> table and figure set a global variable on the specific value
    \let\oldTab=\table
    \renewcommand{\table}{\def\typeOfCap{tab}\stepcounter{anlagenentries}\oldTab}

    \let\oldFig=\figure
    \renewcommand{\figure}{\def\typeOfCap{fig}\stepcounter{anlagenentries}\oldFig}

    \let\oldLongFig=\longfigure
    \renewcommand{\longfigure}{\def\typeOfCap{fig}\stepcounter{anlagenentries}\oldLongFig}
}


% AUTO REMOVE/INSERT Literaturverzeichnis
\newcounter{totalbibentries}
\newcommand*{\listcounted}{}

\makeatletter
\AtDataInput{
    \xifinlist{\abx@field@entrykey}\listcounted
    {}
    {\stepcounter{totalbibentries}
        \listxadd\listcounted{\abx@field@entrykey}}
}
\makeatother


% ACRO SETUP
\acsetup{
    list/heading = section*,
    list/name = {Abkürzungsverzeichnis},
    list/template = description,
    make-links = true,
    link-only-first = false
}

% Standard Abkürzungsverzeichnis überschreiben -> einheitliche Einrückung
\RenewAcroTemplate[list]{description}{%
    \acronymsmapT{%
        \AcroAddRow{%
            \textbf{\acrowrite{short}}%
            &
            \acrowrite{long}%
            \acropages
            {\acrotranslate{page}\nobreakspace}%
            {\acrotranslate{pages}\nobreakspace}%
            \vspace{10pt}
            \tabularnewline
        }%
    }%
    \acroheading
    \acropreamble
    \noindent
    \begin{tabular}{@{}ll}
        \AcronymTable
    \end{tabular}
}

% AUTO REMOVE/INSERT Abkürzungsverzeichnis
% Abkürzungsverzeichnis überschreibt \UseAcroTemplate für \ac
% New Counter to count used acronyms:
\newtotcounter{acro_num}
\def\oldUseAcroTemplate{} \let\oldUseAcroTemplate=\UseAcroTemplate
\def\UseAcroTemplate{\stepcounter{acro_num}\oldUseAcroTemplate}


% INDENTION SETUP
% Abstände und Einrückungen abhängig von config.tex ein-/ausschalten
\if\CEINR 0
    \setlength{\parskip}{6pt}
    \setlength{\parindent}{0cm}
\fi
