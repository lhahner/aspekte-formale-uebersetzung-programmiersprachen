\documentclass[a4paper]{article}
%\usepackage{simplemargins}

%\usepackage[square]{natbib}
\usepackage{amsmath}
\usepackage{amsfonts}
\usepackage{amssymb}
\usepackage{graphicx}

\begin{document}
\pagenumbering{gobble}

\Large
 \begin{center}
Softwaregestüzte Transformation von Legacy Code in moderne Programmiersprachen\\ 

\hspace{10pt}

% Author names and affiliations
\large
Lennart Hahner \\

\hspace{10pt}


\end{center}

\hspace{10pt}

\normalsize
Auf der Welt existieren etwa 7.000 Sprachen; alle zu beherrschen wäre mehr als ein Lebenswerk. Um nicht in völliger Stille zu ersticken, gibt es die Möglichkeit, eine Sprache in eine andere zu übersetzen. Heutzutage ermöglichen computergestützte Lösungen das Verständnis und die Verwendung jeder erdenklichen Sprache. Natürliche Sprachen unterstützen das interpersonelle Verständnis, ebenso wie Programmiersprachen das maschinelle Verständnis fördern. Programmiersprachen schränken die menschliche Sprache stark ein und erlauben nur eine festgelegte Menge an Wörtern, Wortkombinationen und Grammatik.

Um eine Schnittmenge zwischen der maschinenspezifischen Sprache und der komplexen menschlichen Sprache zu finden, entstanden Programmiersprachen. Diese Beschränkung ermöglicht es, unsere Ausdrücke dem Computer durch einen Compiler verständlich zu machen. Da Computer nicht jede menschliche Grammatik verstehen, benötigen wir einen Transpiler zwischen den Programmiersprachen. Dieser transformiert Programmiersprachen, damit sie in eine andere Grammatik injiziert werden können.

Im Jahr 2023 besteht die Möglichkeit, die Grammatikeinschränkung einer Programmiersprache aufzuheben und nur mithilfe natürlicher Sprache Befehle an einen Computer zu erteilen. Large Language Models (LLMs) nehmen natürliche Sprache als Eingabe und erzeugen logische Programmiersprachen als Ausgabe. Im Gegensatz zu Compilern und Transpilern sind LLMs jedoch auf natürliche Sprachen spezialisiert und keine Brücke zwischen Mensch und Maschine, sondern selbst eine Maschine.

Es wird diskutiert, dass es möglich ist, ein Programm nahezu vollständig von einer Programmiersprache in eine andere mit einem LLM zu übersetzen. Dies ist jedoch technisch gesehen fehleranfällig und nicht zuverlässig. Der eigentliche Unterschied zwischen diesen Transformationsmöglichkeiten wird deutlich, wenn der technische Prozess eines Transpilers betrachtet wird. Dabei wird klar, dass es nicht nur darum geht, intuitiv Logik zu übersetzen, sondern dass die Grammatik direkten Einfluss auf die Bedingungen des Computers hat, sei es in Bezug auf Speicherverwaltung, Prozessverwaltung und -steuerung.

\end{document}