%--> Je Subsection Punkt Fragen formulieren.

%-------------------------------------------
% HEADER

% Roterfade der Einleitung:

% 1. Problem -> Kompatibilität
% 2. Ziel -> Lösung mit Übersetzungsprogramm
% 3. Abgrenzung Interpreter & Compiler -> Übersetzungsprogramm als Transpiler
% 4. Formale Grammatike -> Formale Grammatik als Vorrausetzung des Transpilers
% 5. JavaCC -> Implementierung der Formalen Grammtik mit JavaCC und damit des Transpilers

%-------------------------------------------
\pagebreak
\section{Theoretische Grundlagen}
\subsection{Problemstellung}
	
Es gibt zwei Probleme, die eine Übersetzung der \ac{pli} nach Java lösen soll. 

Das erste Problem ist das Portabilitätsproblem von wartungsintensiven PL/I-Programmen, die auf modernen Plattformen wie etwa Cloud-Instanzen oder Linux-Servern laufen sollen. Der PL/I-Compiler, der auf den meisten Computersystemen im Einsatz ist, wird von IBM entwickelt und vermarktet. Hierbei handelt es sich um einen Compiler, der für das von IBM entwickelte Betriebssystem z/OS geschrieben wurde. Eine Kompilierung von PL/I auf einem herkömmlichen x86-Desktop Computer oder einer vergleichbar weit verbreiteten Rechnerarchitektur ist mit diesem Compiler nicht möglich. \footcite[Vgl. ][]{plicomp} Eine Alternative bietet die Organisation GNU mit der \ac{GCC}. Der Softwareentwickler Henrick Sorensen entwickelte Teile des Frontends für einen PL/I-Compiler. Dabei verwendete er das Backend, das die \ac{GCC} zur Verfügung stellt. Jedoch gab es bei diesem Projekt seit 2007 keine weiteren Neuerungen mehr. Der Entwickler gibt an, dass bisher keine Zwischencode-Erzeugung stattfindet, was diesen Compiler bisher unbrauchbar macht. \footcite[Vgl. ][]{pligcc} 
Typischerweise wird heutzutage auf einem Endgerät ein IBM 3270 Terminal emuliert, das eine Verbindung zu einem z/Os-System herstellt, um den PL/I-Code auf diesem zu kompilieren.

Das zweite Problem bezieht sich auf den demografischen Wandel. PL/I ist eine Altsprache, die seit den 1960er Jahren im Einsatz ist und durch den Generationenwechsel an Entwicklern verliert. Wartung und Entwicklung werden so häufig teuer.\footcite[Vgl. ][S. 227ff.]{histopli} Java hingegen ist auf nahezu allen modernen Systemen durch die plattformunabhängige Java Virtual Machine (JVM) kompilierbar. \footcite[Vgl. ][]{jvm} Insbesondere ist eine Kompilierung auch auf einem IBM-Großrechner mit z/OS möglich. \footcite[Vgl. ][]{zos} Das macht Java zu einer flexibel einsetzbaren und beliebten Programmiersprache. \footcite[Vgl. ][]{tiobe} Wiederum führt das zu einer höheren Anzahl an Entwicklern, die in der Lage sind, Java-Programme zu warten.

% Was ist ein Transpiler?
Um die Programmiersprache Java zur Lösung der eingangs beschriebenen Probleme zu verwenden, kann ein Transpiler benutzt werden, der die Transformation des Quellcodes teilweise automatisiert.
Sobald der Programmcode nach Java übersetzt wurde, kann der Java-Zielcode von dem Java-Compiler kompiliert und auf der \ac{jvm} ausgeführt werden.

% @review: Begriffswahl
Diese Arbeit führt im ersten Kapitel in die grundlegenden theoretischen Konzepte wie bspw. Grammatiken von formalen Sprachen ein. Darauffolgend werden weitere Technologie- und Architektur-Entscheidungen des Transpilers vorgestellt. In Kapitel drei werden die Verwendung des Transpilers, die Gestaltung der Übersetzungen, sowie die durchgeführten Tests beschrieben. Abschließend wird in Kapitel vier ein Fazit gezogen und ein Ausblick auf die Weiterentwicklung gegeben.
Durchgehend wird zwischen dem Transpiler, dem Eingabeprogramm (PL/I-Quellcode) und dem Ausgabeprogramm (Java-Zielcode) unterschieden. Diese Begriffe helfen dabei, die verschiedenen Programme eindeutig zu identifizieren. 

%Anderseits auch die veränderte Laufzeit-Performance. Die Laufzeit-Performance kann durch eine Übersetzung verschlechtert, wie auch verbessert werden. Somit ist nicht nur die reine Übersetzung Teil der Problemstellung, sondern es gilt auch die Übersetzung zu beurteilen. 

     

% 	 Welches Problem löst das Programm?
%	 Probleme 
%			 1. Nicht auf jedem System läuft PL/I, besonders nicht auf modernen x86 bzw. Cloud.
%			 2. PL/I ist eine weniger verwendete Sprache, Wartung teuer &  Schwer.

%	 (Hinführung zum Problem:
%	 Historisches Kompatibilitätsproblem -> Nicht auf jedem System lief jede Assambler Sprache, Problem: hoher Aufwand und Unflexibel
% 	 Deshalb -> Compiler mit Hochsprache, der Code für das Backend des Compilers, bspw. C's Gcc Compiler
%    in Assambler Sprache des Systems übersetzt.)? **Hier einen Cross Compiler erklären bzw. im Zusammenhang mit dem Historischen Problem.**

%	 Problem mit PL/I -> Pl/I Compiler rar bzw. nur für Großrechner Systeme vorhanden   
%	 Es gibt zwar einen GCC Pl/I Compiler, dieser wird aber seit 2007 nicht mehr weiterentwickelt. Eine Weiterentwiclung könnte auch Interessant %    sein, löst aber nicht das Problem der teuren Wartung von Programmen in PL/I.

% 	 
%	 Lösungsvorschlag zu 1 -> PL/I zu Java Transpiler bauen Java und JVM relativ System unabhängig und damit Ideale Zielsprache für eine 
%	 hohe Kompatibiltätsrate.Um zum Beispiel Pl/I Programm die auf einem Großrechner laufen auch auf einem x86 On-Prem Server oder einer Cloud
%    zu betreiben. **Hier die Frage klären was ein Transpiler ist**
%	 
%    Lösungsvorschlag zu 2 -> Java ist den großteil der Softwareentwickler bekannt und eine Wartung ist leichter.
%
 \pagebreak
\subsection{Zielsetzung}
% Herleitung von der Problemstellung	
Allgemein soll ein plattformunabhängiger Transpiler entstehen, der die Entwicklung und Transformation von PL/I-Quellcode ermöglicht. Die zugrundeliegende Arbeit stellt die Entwicklung, sowie die Gestaltung der Software dar und diskutiert Gestaltungsentscheidungen. 

% @review: Referenz zur Begriffsklärung
% Zielgruppen Zusammengefasst -> modulare Architektur -> Strategy, Singleton, Composite Pattern   
Diese Arbeit richtet sich zuerst an juniore Anwendungsentwickler in den Sprachen PL/I bzw. Java. Für diese Nutzergruppe soll der Transpiler ein Hilfswerkzeug darstellen. Weiterführend werden diese als \emph{Benutzer} bezeichnet. Eine andere Nutzergruppe sind \emph{Administratoren}, die den Transpiler selbst anpassen und erweitern möchten. Ermöglicht wird dies durch eine modulare Architektur.

% Junior Entwickler, die gerade in PL/1 einsteigen -> erfolgreicher Integrationstest, transpiler übersetzt ganze Programme
Juniore Entwickler profitieren von dieser Arbeit als Einstiegspunkt in die Programmiersprache PL/I. Beispielhaft könnten Entwickler den Transpiler als Test-Umgebung für selbst entwickelten PL/I-Quellcode verwenden. Für Benutzer, die mehr Erfahrung mit Java haben, eignet sich der Transpiler als Lernhilfe. Es wird ihnen so erleichtert, PL/I-Quellcode zu analysieren. Sie können bestehende Kenntnisse aus Java anwenden, um Muster in PL/I wiederzuerkennen. Dies kann den Lernprozess beschleunigen.

% Benutzbarkeit ->  Web-frontend, Einfache Benutzbarkeit, Syntax highlighting, Dokumentation
Der Transpiler aus der Projektarbeit-IV konnte bisher über die Kommandozeile, sowie der IDE Eclipse verwendet werden. Diese ursprüngliche Benutzung des Transpilers führte zu einer hohen Fehleranfälligkeit und Dokumentationsbedarf. Die Komplexität der Benutzung wird durch ein Graphical-User-Inferface (GUI) vereinfacht.
%Das Konzept dieser GUI soll dem eines Übersetzers der natürlichen Sprache, wie etwa 'DeepL' oder 'Google-Translate', ähneln. Mit diesen Konzepten sind Benutzer vertraut, erleichtert den Einstieg in die Programmiersprache PL/I, das Testen des PL/I-Quellcodes sowie die schnelle Übersetzung.

%  Entwickler die das Programm eigenständig erweitern, verändern wollen.
Durch die modularisierte Gestaltung des Transpilers können Administratoren selbst Module austauschen und erweitern. 
Etwa durch eine API-Schnittstelle. 

%  Zusammenfassung und hinführung zum nächsten zu dem Unterschied Interpreter und Compiler ->  Die Variablendeklaration sowie -zuweisung, die Ein- und Ausgabesteuerung und allgemeine Kontrollflussanweisungen wurden im Zuge der Entwicklung implementiert. Als Unit-Tests, dh. Wahrung der Funktionalität ist gewährleistet.

Der Transpiler soll grundlegenden Sprachkonstrukte übersetzen. Die Variablendeklaration sowie -zuweisung, die Ein- und Ausgabesteuerung und allgemeine Kontrollflussanweisungen wurden im Zuge der Entwicklung implementiert. Der entstehende Java-Zielcode ist unter Verwendung einer Codebasis kompilierbar. Hierbei sollen die  PL/I-Konstrukte im Java-Zielcode erkennbar sein, sowie native Java-Strukturen des objektorientierten Programmierparadigmas verdeutlicht werden.

% Aufteilung der Zielstellung:
% 1. Allgemein; Ableitung aus der Problemstellung
% 2. Zielgruppen spezfifisch
% 2.1 Einfache und unkomplizierte Lösung
%
% 2.2 Erweiterung des Transpilers bzw. ersetzen von Modulen	
	
% - Wie eine Art JavaScript Minifier oder 
%  Wer ist die Zielgruppe?
%  - Junior Entwickler die gerade in PL/1 einsteigen.
%   - Lernhilfe
%  - Online-Smoketest von PL/I Code
%   - Benutzbarkeit
%   - Entwickler die das Programm eigenständig erweitern, verändern wollen.
 
%  Ziele der Architektur (Zielgruppe Entwickler)
%  - Perspektive des Entwicklers
%  - Perspektive des Benutzers
    \pagebreak

\subsection{Abgrenzung Interpreter und Compiler}
  
% Wie arbeitet ein Compiler?
Ein Compiler besteht aus einem Frontend und Backend. Das Frontend umfasst die lexikalische, syntaktische, semantische Analyse und die Symboltabelle.
Das Ergebnis des Frontends ist eine Zwischencodedarstellung, die an das Backend übergeben wird, um daraus Maschinencode zu generieren. Der Maschinencode kann auf dem zugrundeliegenden System ausgeführt werden. \footcite[Vgl. ][S.106ff. ]{aho}
Abbildung \ref{fig:compiler} zeigt eine Übersicht und das Ergebnis der einzelnen Compilerphasen. Die Abbildung unterteilt den Ablauf wie eingangs beschrieben in Frontend und Backend.

% @todo: Quelle
\dhgefigure[h]{compiler-ablauf-diagramm.png}[scale=0.6]{Funktionsweise eines Compilers}{fig:compiler}[aho][S. 12]

In der ersten Phase teilt der Compiler den eingegebenen String in Token auf. Danach entsteht ein Syntaxbaum, der in diesem Beispiel die Zwischencodedarstellung repräsentiert. Mithilfe des Syntaxbaums wird der Quellcode auf semantische Fehler überprüft. Ab diesem Punkt beginnt das Backend des Compilers. Zuerst erzeugt das Backend maschinenunabhängigen Code, meist in Assembler und anschließend maschinenabhängigen Code. Die Abhängigkeit besteht im Einsetzen der Register- und Hardwareadressen für die ausführende Maschine.  \footcite[Vgl. ][S.30ff. ]{aho}
In jeder Phase wird aus der Symboltabelle gelesen und geschrieben.

%  Warum ein Transpiler?
Es gibt weitere Lösungen, die auf einem ähnlichen Konzept basieren.
Ein One-Pass-Compiler etwa erzeugt keinen Zwischencode, sondern führt den Code direkt aus. Diese Methode wurde angewendet, um Speicherplatz zu sparen, da frühe Computer nur begrenzte Kapazitäten hatten und keine Zwischenergebnisse speichern konnten. Ein Beispiel für eine Sprache, die mit einem One-Pass-Compiler kompiliert, ist Turbo Pascal. \footcite[Vgl. ][]{onepass}

Eine weitere Ausprägung ist ein Source-to-Source Compiler. Während ein C-Compiler den C-Code nach der Zwischencodeerzeugung in Assemblersprache und anschließend der Assembler den C-Quellcode in Maschinencode übersetzt, wandelt ein Source-to-Source-Compiler beispielsweise C-Quellcode in Java-Zielcode um. 
Ein Source-to-Source-Compiler wird im Zusammenhang dieser Arbeit als \emph{Transpiler} bezeichnet. Abbildung \ref{fig:transpiler} zeigt die Prozessschritte eines Transpilers.

\dhgefigure[h]{transpiler-diagramm.png}[scale=0.6]{Funktionsweise eines Transpilers}{fig:transpiler}[][]

Ein Vergleich von Abbildung \ref{fig:compiler} mit Abbildung \ref{fig:transpiler} zeigt, dass die ersten Phasen bis zur Zwischencodeerzeugung gleich bleiben. In Abbildung \ref{fig:transpiler} sind jedoch die Prozesse des Frontends neben denen des Backends dargestellt, während in Abbildung \ref{fig:compiler} diese untereinander angeordnet sind. Zurückzuführen ist diese Darstellung auf die verwendeten Sprachebenen.

Ein Compiler übersetzt den Quellcode einer Hochsprache in eine maschinennahe Sprache wie Assembler. Hingegen bleibt der Transpiler bei der Ein- und Ausgabe auf der Sprachebene der Quellsprache. In Abbildung \ref{fig:transpiler} werden zudem die Phasen des Backends reduziert, da keine Übersetzung in eine maschinenabhängige Sprache erfolgt.

% @review
Ein weiterer Unterschied ist die Art der Ausführung des übersetzten Zielcodes.
Das Ergebnis der Übersetzung eines Compilers ist Binärcode, der auf einem Zielsystem ausgeführt werden kann. Um hingegen das Ergebnis eines Transpilers auszuführen,
braucht es einen weiteren Compiler. Dieser Compiler muss den Zielcode der Zielsprache in Binärcode übersetzen.
Bei der Übersetzung von PL/I-Quellcode nach Java-Zielcode bedarf es also eines weiteren Java-Compilers zum Ausführen des Zielcodes.

% @review: Cross Compiler? - Hat hier eig nix zu suchen, ist Thema für Boostrapping aber nicht für Transpiler
%Neben den Methoden der Konstruktion, gibt es auch unterschiedliche Verwendungen von Compilern. 
%Etwa existiert der Begriff der Cross-Kompilierung. Hierbei handelt es sich um die Möglichkeit, einen Compiler, der sich auf einem externen Computersystem befindet, zu verwenden, um den Quellcode auf dem lokalen System in Binärcode zu übersetzen. \footcite[Vgl. ][]{guncross}

%Diese Verwendungsweise findet etwa Anwendung beim Bootstrapping. Liegt auf dem System noch kein Compiler für die Sprache vor, in der der Kernel geschrieben wurde, wird diese Methode verwendet, um den Kernel-Code zu kompilieren. 
%In dieser Arbeit kommt ein solcher Ansatz bedingt zum Einsatz. Wird der Transpiler in einem Webinterface verwendet, ist die Verhaltensweise ähnlich.
%Da auch hier der Compiler auf einem anderen Host-Computer, den lokalen Quellcode übersetzt.

Zusammenfassend übersetzen sowohl One-Pass-Compiler, Transpiler bzw. Source-to-source Compiler als auch herkömmliche Compiler das Programm basierend auf einer Zwischencodeerzeugung. \footcite[Vgl. ][S. 18ff. ]{assambly}

Eine weitere Alternative ist der Interpreter. Dieser führt den Quellcode direkt Zeile für Zeile aus, ohne vorher eine Zwischencoderepräsentation zu erzeugen. In Abbildung \ref{fig:shell} ist die Verarbeitung des Bash Interpreters dargestellt.


\dhgefigure[h]{shell_interpreter.png}[scale=0.6]{Ablauf des Bash-Interpreters}{fig:shell}[][]
\pagebreak

Die Eingabe des Bash-Skripts wird zeilenweise gelesen. Das Quoting folgt dem Prinzip der lexikalischen Analyse, bei der alle Sonderzeichen entfernt werden, wie zum Beispiel Kommentare oder Schrägstriche. Sobald das Quoting abgeschlossen ist, entsteht ein String, der nur aus den Token eines Ausdrucks besteht.

Anschließend beginnt das Parsing, das der syntaktischen Analyse im Kompilierprozess ähnelt, jedoch keine Zwischencodeerzeugung beinhaltet. Hier wird lediglich zwischen einfachen Bash-Befehlen wie \verb+wc+ und zusammengesetzten Befehlen wie einem \verb+if+-Ausdruck unterscheiden.

Der Verarbeitungsprozess setzt sich mit der Shell-Expansion fort, bei der in einem Befehl eingebettete Variablen und Pfade durch ihre absoluten Repräsentationen ersetzt werden.
Ab hier verarbeitet nun das Betriebssystem die Befehle des Shell-Skriptes und führt das entsprechende Programm mithilfe der Pfade zur vorkompilierten Binärdatei aus. Dieser Prozess ist vage vergleichbar mit dem Einsetzen von Registeradressen während der Kompilierung.
Schließlich wird das Ergebnis in der Standardausgabe ausgegeben. \footcite[Vgl. ][]{gnubash}

% Absatz: Zusammenfassende Unterscheidung zwischen Interpreter und Transpiler, Was sind gemeinsamkeiten und unterschiede von Transpiler und interpreter?

Zusammenfassend zeigen sich sowohl Ähnlichkeiten als auch Unterschiede zwischen einem Interpreter und einem Compiler bzw. Transpiler. Beide durchlaufen die Phasen der lexikalischen und syntaktischen Analyse, wobei sie den Quellcode zunächst um Kommentare, Leerstellen oder andere für die Übersetzung irrelevante Symbole bereinigen. Anschließend erfolgt entweder die direkte Ausführung des Quellcodes oder die Erzeugung einer unabhängigen Repräsentation.

% Wie sähe mein Programm als Interpreter aus:
Um den PL/I-Quellcode nach Java-Zielcode zu übersetzen, eignet sich ein Interpreter weniger. Da ein Interpreter nur Zeilenweise den Quellcode verarbeitet, wäre auch nur eine Zeilenweise Übersetzung denkbar. 
Jedoch können so die objektorientierten Konstrukte in Java nicht explizit benutzt werden, sondern nur die prozeduralen Bestandteile der Sprache. Dabei bleiben semantische Zusammenhänge in dem PL/I-Quellcode unberücksichtigt, die etwa bei der Transformation von zusammengesetzten Datentypdeklarationen benötigt werden. 
Durch eine Zwischencoderepräsentation in Form eines Syntaxbaums kann der Kontext leichter berücksichtigt werden. Dadurch kann die Qualität des Java-Zielcodes erhöht werden.

%Um PL/I-Code korrekt in Java zu übersetzen, sind Verbindungen zwischen den Ausdrücken relevant. Diese Verbindungen können in einer Zwischencoderepräsentation des PL/I-Quellcodes, komprimiert dargestellt werden. Eine zeilenweise Übersetzung könnte zu einem Java-Programm führen, das den restlichen Kontext des Programms nicht weiter berücksichtigt.
%Ein Interpreter erreicht das Ziel einer objektorientierten Programmierung und eines wiedererkennbaren PL/I-Codes nicht. Verschachtelte Strukturen lassen sich schwer in Klassen umwandeln. Ein rekursiver Syntaxbaum unterstützt diese Transformation, während ein Interpreter nur statische Umsetzungen ermöglicht. Dies führt zu mehr Boilerplate-Code und geringerer Lesbarkeit. Daher ist ein Interpreter ungeeignet.
%Da eine rekursive Verarbeitung, wie es etwa mit einem Syntaxbaum möglich ist, so nicht möglich ist. 

% Der entscheidende Unterschied liegt darin, dass der Interpreter den Quellcode lediglich zeilenweise direkt übersetzt, während der Compiler das Programm in eine andere Form transformiert und liest. Dabei stehen die verwendeten Ausdrücke des Eingabecodes in Beziehung zueinander, beispielsweise durch die Verschachtelung von Verzweigungen und Schleifen.



% - Erweiterung des Umfangs während der Laufzeit
% - Trennung Laufzeit/Konzeptionsphase

% - Hier erwähnen das eine geminsamkeit die definition von Grammatiken ist, dann überleiten zu Formale Grammatiken.
% - Auch Entscheidung treffen was genau der Transpiler ist, Compiler oder Interpreter
\pagebreak
   
   
\subsection{Formale Sprachen und ihre Grammatiken}
% Formale Sprachen - linearer Ausdruck
% Linear -> Array von Wörtern
% 
In diesem Kapitel wird das Vorgehen der syntaktischen und lexikalischen Analyse vorgestellt. Das Schreiben und Lesen von Sprachen ist sowohl für Menschen als auch für Computer ein linearer Prozess. Für diesen Prozess würde es ausreichen, den Tokenstrom in einer Liste bzw. Array zu verarbeiten.

Um einen Text zu verstehen, sind komplexere Strukturen erforderlich, um zum Beispiel Satzzusammenhänge, Referenzen und Kontexte aufzulösen. Solche Strukturen werden in der Sprachforschung seit den 1960er Jahren durch generative Grammatiken beschrieben. \footcite[Vgl. ][S. 117]{choms} \footcite[Vgl. ][S. 149ff. ]{automata}
Eine Grammatik wird dabei durch ein Tupel \emph{G} beschrieben:

\begin{center}
	\begin{equation}\label{eqn:grammar}
		G=(V,T,S,P)
	\end{equation}
\end{center}


Hierbei steht \verb+V+ für Variablen bzw. Nichtterminalsymbole, \verb+T+ für Terminale, \verb+S+ für Start und \verb+P+ für Produktionsregeln. Dabei ist \verb+S+ ein Teil von \verb+V+. Produktionsregeln sind gleichbedeutend mit Syntaxregeln von Programmiersprachen. 
Der Prozess, einen Tokenstrom aufgrund von Syntaxregeln zu verarbeiten, nennt sich Parsen.
Hier werden die Ketten von Wörtern in eine Baumstruktur transformiert.
Die Baumstruktur ergibt sich aus der Reihenfolge der Anwendung der Syntaxregeln. In der Gleichung 2 bis 6 ist die Syntaxregel zur Erzeugung einer einfachen \verb+if+ und \verb+else+ Verzweigung dargestellt.

% @todo: PL/I Verwenden und Kontext herstellen
\begin{center}
	\begin{equation}\label{eqn:start}
		S \to \mathbf{if}\: expr\: \mathbf{then}\: stmt\: \mathbf{else}\: stmt\: | \mathbf{if}\: expr\: \mathbf{then}\: stmt;
	\end{equation}
	\begin{equation}
		expr \to expr\: op\: term\: | term
	\end{equation}
	\begin{equation}
		op \to \mathbf{>}\: |\: \mathbf{<}\: |\: \mathbf{=}\: |\: \mathbf{!}
	\end{equation}
	\begin{equation}
		term \to term\: multOp\: factor\:
	\end{equation}
	\begin{equation}
		factor \to \mathbf{id}\: |\: \mathbf{constant}
	\end{equation}
\end{center}

\pagebreak
Mit der beschriebenen Grammatik ist der PL/I-Quellcode in Listing \ref{lst:pliifstatement} zulässig.

\begin{lstlisting}[language=PL/I, caption=PL/I-Verzweigung, label={lst:pliifstatement}]
	IF A > B THEN
	CALL proc_1;
	ELSE
	CALL proc_2;
	END
\end{lstlisting}

Nicht zulässig ist hingegen der Ausdruck in Listing \ref{lst:pliwrongstatement}.

\begin{lstlisting}[language=PL/I, caption=Ungültiges PL/I, label={lst:pliwrongstatement}]
	CALL IF THEN A > B proc_1;
\end{lstlisting}

Je nach Umfang der Symbole und Syntaxregeln einer Sprache kann ein Parser sehr komplex sein. 
Durch die rekursive Natur der Produktionsregeln können bereits wenige Regeln eine Vielzahl an möglichen Ausdrücken erzeugen.
Ein weiterer Grund dafür sind die unterschiedlichen Kombinationsmöglichkeiten von Regeln, die das Programm berücksichtigen muss. Die Grammatik von PL/I kann Mehrdeutigkeiten aufweisen, welche im Compiler aufgelöst werden müssen. \footcite[Vgl. ][S. 262ff. ]{compibau}

Aus diesen Gründen wird in der Regel ein Parser mit einem Compiler-Compiler erzeugt.
Der Entwickler spezifiziert eine Grammatik in einer formalen Darstellung, woraufhin der Parser aus dieser automatisiert erzeugt werden kann. Der gesamte Ein- und Ausgabeverarbeitungsprozess ist in Abbildung \ref{fig:ausgabe} dargestellt.


\dhgefigure[h]{parser-generation.png}[scale=0.65]{Ein- und Ausgabeverarbeitungsprozess des Compiler-Compiler}{fig:ausgabe}[][]


% Hier Grund aus Compilerbau Buch


% TODO Warum 
% Warum braucht ein Compiler eine Grammatik?
% In dem vorangegangen Kapitel wurden unterschiedliche Methoden, Anwendungsgebiete und Formen der Sprachinterpretation eines Computers vorgestellt. Damit die Interpretation von Sprachen korrekt erfolgt, braucht ein Computer Regeln. Grammatiken beinhalten diese Regeln. 

% - Theoretischer Abriss
% - Einordnung der resultate der PA 4
%- PL/1 Syntax Wo zu finden? Welche Version der Sprache? 
%	- Reguläre Ausdrücke Syntax, Beispiel einer Grammatik die ich mit verwende, Typ einer Grammatik
%  - Literatur
%    - Chomsky Hierarchie Bücher
%    - https://www-igm.univ-mlv.fr/~berstel/LivreCodes/Codes.html
%  - Woraus besteht eine Grammatik?
%   - Wie lassen sich Grammatiken der Komplexität nach anordnen?
%  - Chomsky Hierarchie
%  Erst Chomsky Hierachier, dann nach Komplexität einordnen und am Beispiel von Regulären Ausdrücken und PL/I Grammatik einführen.
     
\pagebreak
\subsection{Anwendung von formalen Grammatiken in JavaCC}
% 1. Was ist ein Compiler-Compiler? (Verbindung von formalen Grammatiken zu JavaCC)

Ein Compiler-Compiler wie JavaCC ist eine Technologie, mit der aus einer formalen Beschreibung einer Grammatik ein Lexer und ein Parser erzeugt werden. 

Der Lexer und Parser wenden die in der Grammatikdatei definierten Regeln an und verarbeiten in einer Java-Klasse die übergebenen Ausdrücke.
Die Darstellung der Grammatik erfolgt in einer Art \ac{ebnf}.  Beispielhafte Compiler-Compiler sind neben JavaCC Yacc, Antlr und Lexer. In Kapitel 1.4 wurde bereits vereinfacht eine Syntaxregel aus der PL/I-Grammatik dargestellt. Ähnlich erfolgt auch die Darstellung in einer Grammatik-Datei. Folgendes Beispiel zeigt die Darstellung eines \verb+IF ELSE+ Ausdrucks. 


\begin{lstlisting}[language=Java, caption=JavaCC Syntaxregel einer Verzweigung, label={lst:ifstatement}]
	void if_statement() #BRANCH :
	{}
	{
		< IF >bool_expression()
		< THEN >proc_body()
		[else_statement()]
	}
\end{lstlisting}

In JavaCC besteht die Möglichkeit, die Beschreibung von Syntaxregeln in Methoden zu Kapseln.
In Zeile 1 ist der Methodenkopf zu sehen. In diesem Fall erzeugt die Methode keinen Rückgabewert.
Das durch die Raute gekennzeichnete Symbol ist die Repräsentation im Syntaxbaum und zusätzlich auch die Darstellung im Zwischencode.
Im Körper der Methode wird der If-Ausdruck weiter definiert. Terminalsymbole werden in der JavaCC Grammatik mit den größer-als und kleiner-als Zeichen eingeklammert. 

Wie in Listing \ref{lst:ifstatement} dargestellt, sind die Nichtterminalsymbole wiederum Methoden, die  weitere Ausdrücke beschreiben. Die Methode \verb+bool_expression+ beschreibt einen zulässigen booleschen Ausdruck, der durch einen booleschen Operator mit einem weiteren booleschen Ausdruck verknüpft werden kann. Weiterhin wird in der Methode \verb+proc_body+ beschrieben, welche Ausdrücke weiter zulässig sind. Dazu zählt bspw. auch eine weitere \verb+IF ELSE+ Verzweigung. Ähnlich wird bei der Syntaxregel \ref{eqn:start} so lange reduziert, bis lediglich Terminalsymbole übrig bleiben.

So werden aus der Grammatikdatei, durch den Compiler-Compiler Java-Klassen erzeugt, die Programmroutinen zur syntaktischen Verarbeitung von PL/I-Ausdrücken beinhalten.
Der Ausschnitt aus der Grammatik-Datei in Listing \ref{lst:ifstatement} wird zu dem Java-Quellcode in Listing \ref{lst:ifstatementmethode}.
\pagebreak

\begin{lstlisting}[language=Java, caption=Syntaxregel für Verzweigungen als Java-Methode, label={lst:ifstatementmethode}]
	final public void if_statement() throws ParseException {
		SimpleNode jjtn000 = new SimpleNode(JJTBRANCH);
		boolean jjtc000 = true;
		jjtree.openNodeScope(jjtn000);
		try {
			jj_consume_token(IF);
			bool_expression();
			jj_consume_token(THEN);
			proc_body();
			
			if (jj_2_30(3)) {
				else_statement();
			} else {
				;
			}
		} catch (Throwable jjte000) {
			if (jjtc000) {
				jjtree.clearNodeScope(jjtn000);
				jjtc000 = false;
			} else {
				jjtree.popNode();
			}
			if (jjte000 instanceof RuntimeException) {
				if (true) throw (RuntimeException)jjte000;
			}
			if (jjte000 instanceof ParseException) {
				if (true) throw (ParseException)jjte000;
			}
			if (true) throw (Error)jjte000;
		}
		//Rest ausgeschnitten
	}	
\end{lstlisting}

In Listing \ref{lst:ifstatementmethode} ist zu erkennen, dass die ebenfalls definierten Methoden in der Grammatikdatei, die repräsentativ für die Nichtterminalsymbole sind, auch in der Methode der generierten Parser-Klasse zu Methodenaufrufen führen. Dies ist in Zeile 7 zu sehen. Weiterhin werden Token, also die Repräsentationen der Terminalsymbole, durch die \verb+jj_consume_token+ verarbeitet. Die restlichen Verzweigungen in der generierten Java-Methode prüfen die verarbeiteten Token auf Fehler. Außerdem wird mit dem Objekt \verb+SimpleNode+ in der Java-Methode aus Listing \ref{lst:ifstatementmethode} ein Knoten im Syntaxbaum erzeugt. Diese Knoten werden von den Übersetzungs- \newline modulen des Transpilers verarbeitet.
% Warum ein Compiler-Compiler verwenden?

\pagebreak
Grammatiken für Cobol und Java sind durch die JavaCC Community öffentlich zugänglich. Im Fall von PL/I ist keine Grammatik bekannt, weshalb bei der Entwicklung des Transpilers die IBM Language Reference für PL/I die Hauptquelle für die Grammatikdatei ist.  \footcite[Vgl. ][]{javaccdoku}

% Wie wird JavaCC in die Entwicklung des Transpiler eingebunden? (Hinleitung zur Architektur beschreibung)
Durch die Verwendung des JavaCC \verb+jjtree+-Moduls kann global innerhalb des Projekts auf den Syntaxbaum zugegriffen werden. 
Der Syntaxbaum wird unter anderem durch die Module verwendet, die die semantische Analyse und Synthese durchführen.
Erst durch diese wird der Java-Code erzeugt.
Die gesamte Architektur des Transpilers und dessen weitere Module wird im nachfolgenden Kapitel betrachtet. 
\pagebreak
\section{Architektur}
\subsection{Verwendete Technologien}
%- Compiler Compiler -> JavaCC: Integration in das Projetk, Grund für die Wahl der Technologie
%In Kapitel 1.4 wurde bereits mit JavaCC eine verwendete Technologie vorgestellt. In diesem Kapitel werden die weiteren verwendeten Technologien vorgestellt.

%- Programmiersprache -> Java: Integration in das Projekt, Grund für die Wahl.
Der Transpiler wurde in der Programmiersprache Java, der Version 17 von Oracle entwickelt. 
Java eignet sich als objektorientierte Hochsprache gut für die Entwicklung des Transpilers, da so eine lose Kopplung der Klassen realisiert werden kann und damit eine modulare Bauweise des Softwareprojekts. Weiterhin zählen die Vorteile, die in Kapitel 1.1 für Java erwähnt wurden ebenso in diesem Fall.  Als Compiler-Compiler fiel die Wahl auf JavaCC. Ein alternativer Compiler-Compiler für Java ist Antlr. 

%- IDE -> Eclipse: Integration in das Projekt, Grund für die Wahl.
Der Java-Quellcode des Transpilers wurde in Eclipse geschrieben. Die \ac{ide} Eclipse ermöglicht eine kostenlose Entwicklung, Verwaltung, Überprüfung und Kompilierung von Java Software-Projekten. Weiterhin bietet Eclipse ein breites Software-Repository an Plugins, um die Funktionalität der  \ac{ide} zu erweitern. Dadurch ist eine Integration von JavaCC in die  \ac{ide} möglich und erleichtert die Entwicklung des Parsers.

%- Maven -> Dependency Management: Integration in das Projekt, Grund für die Wahl der Technologie
Es wurde das Software-Projektmanagement-Werkzeug Maven eingesetzt. Maven löst mithilfe des \ac{pom} Abhängigkeiten. Dadurch werden Benutzer und Administratoren beim Build-Prozess entlastet.

%- Testing -> JUNit Tests: Integration in das Projekt, Grund für die Wahl der Technologie
Zum Testen der Anwendung wurde das Java-Test-Framework JUnit 5 verwendet. Es wurden für ausgewählte Klassen zugehörige Testklassen geschrieben. In den Testklassen wurden komplexere Methoden isoliert getestet. Die Wahl von JUnit ist begründet durch die einfache Handhabung, die Kompatibilität mit Eclipse und Maven. Weiterhin ist JUnit eines der bekanntesten Unit-Test-Frameworks für Java-Quellcode. Alternativen sind  TestNG \footcite[Vgl. ][]{testng} oder Mockito. \footcite[Vgl. ][]{mockito}

%- Platform -> Spring
Für die Entwicklung des Frontends wurde das Java-Framework Spring-Boot verwendet.
Dabei wurde das Web-Interface mit der CSS-Softwarebibliothek Bootstrap gestaltet.
Spring-Boot wird für das abhandeln von REST-Anfragen verwendet und bindet über eine API-Schnittstelle den Transpiler ein. Dadurch wird eine Interaktion über den Browser möglich. Darüberhinaus hat der Administrator alternative Übersetzungsschnittstellen einzubinden. Denkbar wäre hier der Einsatz eines \ac{llm}s.

Die Entwicklungsphase folgte jedoch nach der Konzeptionsphase, in der die Architektur der Anwendung ausgearbeitet und später als Quellcode realisiert wurde. 
Im nächsten Kapitel wird dieses Architekturbild vorgestellt.

\pagebreak
\subsection{Module des Transpilers} 

Abbildung \ref{fig:modules} zeigt eine Übersicht aller Module des Transpilers.

\dhgefigure[h]{AbstraktesUML_1.png}[scale=0.5]{Aufbau des Transpilers}{fig:modules}[][]

% Wie sind die Module momentan gebaut?
% App Modul
Die Verarbeitung des PL/I-Quellcodes beginnt mit dem App-Modul. Das App-Modul ist die Schnittstelle für alle weiteren Module. Ein Modul kann durch die Instanziierung der Hauptklasse des zugehörigen Moduls eingefügt werden. Entfernt wird das Modul durch das Löschen der Instanz. Das App-Modul beinhaltet auch die \verb+main+ Methode und ist somit auch der Startpunkt der \ac{jvm}.

Der Scanner wird als erstes instanziiert. Dieser liest aus der Konfigurationsdatei den Pfad der zu übersetzenden PL/I Datei. Die Datei wird als \verb+InputStream+ an den Parser übergeben.
Der durch JavaCC erzeugte Parser wird ebenfalls im App-Modul instanziiert. Dieser behandelt den PL/I-Quellcode entsprechend der vorher definierten Grammatik.  Während des Parsings werden Variablen-, Prozeduren- oder Packagebezeichner des PL/I-Quellcode in die Symboltabelle eingefügt. Das Ergebnis des Parsers ist ein Syntax-Baum. 
In Abbildung \ref{fig:parsetree} ist exemplarisch ein Syntaxbaum des Transpilers abgebildet, der eine Variablendeklaration und die Definition einer Prozedur darstellt.

\dhgefigure[h]{parsetree-example.drawio.png}[scale=0.75]{Beispielhafter Syntaxbaum des PL/I-Parsers}{fig:parsetree}[][]
\pagebreak
Ist der Syntaxbaum entsprechend erzeugt, wird dieser durch das Checker-Modul weiterverarbeitet. In diesem Modul wird die semantische Analyse des Quellcodes durchgeführt. In der aktuellen Version erfolgt eine Typüberprüfung der initialisierten Variablen.

% @review: Hier erwähnen wie Configdatei bzw. Scanner den Ausgabe Ordner beeinflussen wenn das implementiert wurde.
Das Mapper-Modul repräsentiert die Synthese nach Java. Der Syntaxbaum wird dazu Knoten für Knoten abgearbeitet und mit entsprechenden Java-Ausdrücken übersetzt.

% @todo: Wie sind Module zueinander abhängig?
Um einen Überblick über die Abhängigkeiten der Module zu verschaffen, zeigt
Abbildung \ref{fig:modulesdep} die Beziehungen der Module untereinander.

\dhgefigure[h]{Beziehungen_Modules.png}[scale=0.65]{Die Abhängigkeiten der Module}{fig:modulesdep}[][]


In Abbildung \ref{fig:modulesdep} ist zu erkennen, dass die Abhängigkeiten unter den Modulen eine kaskadierende Form aufweisen. Diese Form hat den Vorteil, dass Module aus dieser Kette verändert werden können, ohne vertikal verlaufende Module direkt zu beeinträchtigen. Dennoch ist an dieser Stelle zu erwähnen, dass kein Modul isoliert betrachtet werden darf. Denn eine Veränderung eines Moduls bedeutet, dass auf horizontaler Ebene die Verarbeitung verändert wird. Jedes Modul gibt die Ergebnisse in Pfeilrichtung weiter, bis Zielcode oder eine Fehlermeldung entsteht.

% @todo: Wie wird es erweitert?
Der Administrator kann die Module erweitern oder ersetzen. Dabei sollte jedoch die in Abbildung \ref{fig:modulesdep} dargestellte Kaskade der Module untereinander berücksichtigt werden. Entscheidet sich der Administrator dazu, etwa eine Methode zu entfernen und ein selbst entwickeltes zu verwenden, ist lediglich die bisherige Referenz zu ersetzen.
%In der vorherigen Version des Transpilers wurde etwa die Lexer-Klasse entfernt und die Klassen des Compiler-Compilers verwendet.
%Der Quellcode des Lexer besteht jedoch und ist lediglich als Deprecated markiert. Die erneute Verwendung des Lexer würde über den Aufruf der Methode in Main erfolgen.
%Hingegen wären hier weitere Schritte notwendig, wie etwa das einfügen einer temporären Datei, die von dem Parser als InputStream entgegen genommen wird und weiter verarbeitet wird.

Nachdem nun ein Überblick über die Architektur gegeben wurde, soll in den folgenden Kapiteln eine Detailansicht des Transpilers erfolgen. Ab diesem Punkt soll genauer auf den Quellcode der einzelnen Module eingegangen werden, um nachzuvollziehen, wie der Transpiler den PL/I-Quellcode in Java-Zielcode umwandelt.

%Bausteine
%- Software Architektur
	%- Planen mithilfe eines UML
	%- UX Design 
    %    - zweite Diagramm, des Benutzerfluss
    %    - wie Benutzung abläuft
	%	- Website?
	%	- Docker Container?
%- Fehlertracking
%- Struktuierung des Programms, sodass ein Benutzer es selbständig erweitern kann

%\subsection{Aspektorientierte Programmierung}
%- Wie funktioniert Aspektorientiert Programmierung?
%	- Wie Löse ich mit Aspektorientierter Programmierung konkrete Probleme?
%- JavaBeans
%- Spring
%	- Wie setzt Spring Aspektorientierte Programmierkonzepte ein?


%Wie in dem vorangegangen Kapitel schon dargestellt, werden in jedem Modul
%die Verarbeitungsschritte aus Abbildung \ref{fig:transpiler} implementiert.
%In diesem Kapitel sollen die Module nach ihrer Verarbeitungsreihenfolge vorgestellt werden.
%Dabei soll ein \ac{uml}-Diagramm je zu beginn der Unterkapitel verdeutlichen wie die Klassen
%in den Modulen zueinander aufgebaut sind. In jedem Diagramm wird auch auf die Einbindung in die App-Klasse eingegangen.
%Die Beschreibung beginnt mit dem Scanner.

\subsubsection{Der Scanner und Parser}
\paragraph*{Scanner}
Wie in Kapitel 1.5 eingeführt, wird der Parser vollständig durch JavaCC generiert. 
Damit der Parser den PL/I-Quellcode in die Zwischencodedarstellung übersetzen kann,
braucht dieser eine PL/I-Datei, die als Input-Stream übergeben wird.
Den Pfad zu der Datei gibt der Administrator in der Konfigurationsdatei an.
Das Scanner-Modul liest die Konfigurationsdatei ein und versorgt den Parser mit notwendigen Ressourcen.
Abbildung \ref{fig:scannermodul} zeigt das UML-Diagramm des Scannermoduls.

\dhgefigure[h]{scanner-klasse-uml.drawio.png}[scale=0.68]{Das Scanner-Modul}{fig:scannermodul}[][]

Das Scanner-Modul besteht aus der Klasse \verb+InputReader+. 
Die \verb+InputReader+ Klasse liest und verarbeitet die Konfigurationsdatei. Die Methode \verb+getInputFilePath+ gibt den Pfad der PL/I-Quellcodedatei als String zurück.
Dieser String wird beim Aufruf der \path{getInputFile} Methode benötigt, um aus der Datei einen \verb+InputStream+ zu erzeugen.
Mithilfe des InputStreams kann die Datei als Parameter an den Parser in der Klasse \verb+App+ im Modul \verb+App+
übergeben werden. 

%\paragraph{Lexer}
%Die Klasse Lexer des Moduls Scanner enthält den ehemals selbst geschriebene Lexer für die lexikalische Analyse. Dadurch das dieser Prozess nun vollständig von dem JavaCC-Parser übernommen wird, wurde die Methode \verb+getToken+ überflüssig. Diese ist als veraltet mit der Kennung 'Deprecated' beschrieben, sie kann im Projekt noch verwendet werden, jedoch mit einem Risiko das die Ergebnisse der Methode nicht korrekt sind. 
%Zu einem späteren Zeitpunkt ist denkbar den selbstgeschriebenen Lexer zu optimieren und erneut einzubinden. Weshalb dieser nicht gelöscht wurde. 
%Weitere Methoden in dieser Klasse werden ebenfalls nicht länger von Klassen aus anderen Modulen verwendet. 

\paragraph*{Parser}
Das Parser-Modul deckt die lexikalische und syntaktische Analyse des Quellcodes ab. Wie schon in Kapitel 1.4 erwähnt, werden Klassen des Parsers durch die \verb+.jjt+ Grammatikdatei generiert. Da diese Klassen sehr umfangreich sind, werden diese in Abbildung \ref{fig:moduleparser} lediglich in abgekürzter Form dargestellt. 

\dhgefigure[h]{parser-module-uml.drawio.png}[width=160mm,height=122mm]{UML des Parser-Moduls}{fig:moduleparser}[][]


Der Parser wird über die Klasse \verb+Pl1Parser+ im App-Modul instanziiert. In der Parser-Klasse sind jegliche manuell geschriebenen Methoden aus der Grammatikdatei integriert. 
Dazu gehört etwa die Methode \verb+installId+. Die Klasse \verb+Pl1Parser+  beinhaltet auch die Methoden für die Verarbeitung der Nichtterminalen Ausdrücke, wie zum Beispiel die Methode \verb+if-statement+. Hier ist auch der Großteil der Verarbeitungslogik des Parser-Moduls implementiert. Wird ein Ausdruck während der Verarbeitung in der Pl1Parser-Klasse als nicht zulässig  identifiziert, wird eine \verb+ParseException+ geworfen, die in der Klasse \verb+ParseException+ definiert ist. Die ebenfalls in der Grammatikdatei definierten Token, werden in \path{Pl1ParserConstants} als Konstanten definiert. Während der lexikalischen Analyse wird die Klasse Token verwendet, um Terminalsymbole zu verarbeiten.

Ein Knoten des Syntaxbaums wird durch ein Objekt der Klasse \verb+SimpleNode+ erzeugt. Durch die Methoden \verb+clearNodeScope+ und \verb+closeNodeScope+ wird der verarbeitete Ausdruck mit bspw. der Repräsentation \verb+VAR+ in den Syntaxbaum eingefügt. Hat der Parser ein Nichtterminalsymbol gefunden, wird die Methode \verb+installIds+ aufgerufen und der Bezeichner unter bestimmten Bedingungen in die Symboltabelle eingefügt. Ein Bezeichner wird nicht eingefügt, wenn dieser schon vorhanden ist und denselben Sichtbarkeitswert hat. Dabei wird das Modul \verb+SymbolTable+ implementiert und mit der Methode \verb+insertId+ ein Bezeichner in die Symboltabelle eingefügt. Im folgenden Unterkapitel wird das Modul der Symboltabelle genauer beschrieben. 

\subsubsection{Symboltable}
Die Symboltabelle speichert die PL/I-Symbole in einer Hashtabelle ab. Mit ihr soll bspw. der Parser
erkennen können, ob es sich bei dem ausgewählten Token um ein PL/I-Symbol handelt, oder um einen Bezeichner.
In Abbildung \ref{fig:symboltable} ist das Modul \verb+symboltable+ in \ac{uml} dargestellt.

\dhgefigure[h]{symboltable-module-uml.drawio.png}[scale=0.6]{UML des Symboltabellen-Moduls}{fig:symboltable}

Das Modul \verb+Symboltable+ enthält die Klasse \verb+SymbolTable+ sowie die Enums Pl1Symbols und Template. Die Klasse \verb+SymbolTable+ wird beim instanziieren, mit den Werten aus dem Enum Pl1Symbols initialisiert. Von der Symboltabelle soll während der Laufzeit nur eine Instanz existieren. Erreicht wird dies durch das Singleton-Entwurfsmuster.
Dazu wird in der Klasse Symboltable die Instanz symbols erzeugt. Falls diese null ist, ist eine Instanziierung möglich. Falls sie hingegen bereits instanziiert ist, wird lediglich die Instanz zurückgegeben. Über die Methode \verb+getInstance+ können andere Module auf diese Instanz zugreifen.

Um einen Bezeichner in die Symboltabelle einzufügen, wird ein String-Array mit dem Bezeichner, dem Typ, der Sichtbarkeit und Hierarchiestufe der Variable benötigt. Die restlichen Methoden der Klasse \verb+SymbolTable+ dienen der dezidierten Abfrage von Werten. Bspw. wird die Methode \verb+getSymbolModifier+ dazu verwendet, den Sichtbarkeitswert eines Bezeichners auszugeben.

% Wie funktioniert Template?
Der weitere Enum \verb+Template+ dient im Prozessschritt der Synthese dazu, den Syntaxbaum in Java-Zielcode umzuwandeln.
In diesem Enum sind Java-Quellcodeausschnitte hinterlegt, die PL/I-Quellcode repräsentieren können. Sie werden in dem Mapper-Modul verwendet, um den Java-Zielcode zu erzeugen.
Um den Syntaxbaum zu verwenden, muss der Syntaxbaum Knoten für Knoten überprüft werden. Dieser Prozess erfolgt im Checker-Modul.
 
\subsubsection{Checker}
Das Checker-Modul prüft die Semantik des PL/I-Quellcodes und ist damit repräsentativ für die semantische Analyse. 
Eine Aufgabe des Checker-Moduls ist es, die Typ-Definition von Variablen mit den Zuweisungen abzugleichen.
Wird beispielsweise dem Decimal-Typ ein alphanumerischer Wert zugewiesen, wird dieses Modul den Benutzer auf einen semantischen Fehler hinweisen. Um solche Fehler zu erkennen, wird das Checker-Modul nach dem Composite Entwurfsmuster implementiert. Dadurch können Klassen, die das \verb+ITypeExpression+-Interface implementieren, einheitlich über das Kompositum \verb+VarChecker+ behandelt werden.
In Abbildung \ref{fig:checker} ist das \ac{uml} des Checkermodul abgebildet.
\pagebreak
\dhgefigure[h]{checker-module-uml.png}[scale=0.55]{UML des Checker-Moduls}{fig:checker}

Die Methode \verb+getType+ erschließt die korrekte Typzuweisung von Variablen. Mit diesem Entwurfsmuster kann die Typprüfung um weitere Speicherstrukturen ergänzt werden, bspw. Arrays, Listen oder andere.

In dem Checker-Modul wird auf den Syntaxbaum, das Ergebnis des Parser-Moduls, zugegriffen.
Um auf alle Knoten des Syntaxbaums zuzugreifen, wurde in der \verb+Checker+ Klasse die Methode \verb+iterateTree+ als Tiefensuche rekursiv implementiert.
Die Iterationsweise des Algorhitmus mit dem Syntaxbaum ist in Abbildung \ref{fig:parsetreealgo} dargestellt.

\dhgefigure[h]{parsetree-example-searchalgo.png}[scale=0.465]{Beispielhafte Iteration des Syntaxbaums}{fig:parsetreealgo}[][]

So wird es möglich, gezielt nach Knoten zu suchen und die Attribute des Knotens auszulesen.
Im Checker Modul wird dieser Algorithmus vor allem benutzt, um festzustellen, welche Variablen mit welchem Typ und welchem Bezeichner deklariert wurden und ob dieser Bezeichner in einem \verb+ASSIGN+ Knoten wiederverwendet wurde. So kann der Syntaxbaum mit seinen Attributen vollständig überprüft werden.

Die Überprüfung ist dann abgeschlossen, wenn die Iteration des Suchalgorithmus wieder beim Ausgangspunkt angekommen ist. In Abbildung \ref{fig:parsetreealgo} ist das der Knoten \verb+PRORGRAM+. Wurde kein Fehler entdeckt, beginnt die Transformation nach Java. Falls jedoch ein Fehler entdeckt wurde, wird eine entsprechende Fehlermeldung angezeigt. Das Checker-Modul bereitet den Syntaxbaum entsprechend so vor, dass dieser weiterhin in der Synthese verarbeitet werden kann. Dieses Modul erwartet aktuell eine einfache Quellcode-Programmstruktur. Es fehlen Unit-Tests zur Wiederverwendung von Variablennamen mit unterschiedlichen Typen.

\subsubsection{Mapper}
Das Mapper-Modul transformiert die Zwischencoderepräsentation nach Java. Dazu wurde das Mapper-Modul nach dem Strategie Entwurfsmuster gestaltet.
In Abbildung \ref{fig:mapper} ist das UML des Mapper-Moduls zu sehen. 
Die Mapper-Klassen stellen die Strategien dar. Sie implementieren alle das \path{ITranslationBehavior}-Interface und aufgrund dessen eine \verb+translate+ Methode, die den übersetzten Ausdruck zurückgibt. Dadurch wird die Verwendung verschiedener Algorithmen zur Laufzeit ermöglicht. 

\dhgefigure[h]{mapper-module-uml.png}[scale=0.6]{UML des Mapper-Moduls}{fig:mapper}
\pagebreak
Die Klasse \verb+Mapper+ wird in der Main-Methode des Projektes instanziiert. Als Parameter wird der Wurzelknoten des Syntaxbaums übergegeben.
In der \verb+iterateTree+ Methode wird jeder Knoten überprüft, ob eine solche Mapper-Klasse in der HashMap instanziiert wurde.
Jeder Knoten wird über eine Konstante mit einer zugehörigen Identifikationsnummer überprüft, die in dem Interface \verb+Pl1ParserTreeConstants+ des Parser-Moduls definiert sind. 
In der Klasse \verb+AstMapper+ sind jegliche Identifikationsnummern der Knoten zu finden. Hier werden diese mit den zugehörigen Mapper-Klassen in einer HashMap gespeichert.

Falls eine Klasse vorhanden ist, wird in der TranslationBehavior-Klasse das entsprechende Strategie-Objekt des Typs \path{ITranslationBehavior} in der
Mapper-Klasse gesetzt. So wird die Translate-Methode der aktuell gesetzten Klasse aufgerufen und der Knoten des Syntaxbaums nach Java übersetzt.

% wie funktioniert der PicturemMapper
Eine exemplarische Mapper-Klasse ist etwa die \verb+PictureMapper+ Klasse. Diese enthält Zeichenkettenbeschränkungen des PL/I-Picture Typs und deren Übersetzung als regulären Ausdruck.
Mit der \verb+getRegex+ Methode der \verb+PictureMapper+ Klasse wird der übersetzte reguläre Ausdruck als String zurückgegeben.
Der PL/I-Ausdruck \verb+(4)A+ wird etwa zum regulären Ausdruck \verb+[A-Za-z ]{4}+.   

Neben der PictureMapper Klasse erzeugen weitere Mapper Klassen Knoten für Knoten den Java-Zielcode. 
Alle Ausdrücke werden in einer Arraylist mit dem Namen \verb+javaExpression+ der Mapper-Elternklasse  gespeichert.
Wenn über jeden Knoten des Syntaxbaums iteriert wurde, ist das Programm vollständig übersetzt. Dabei kann es jederzeit zu Fehlern durch eine fehlerhafte Benutzereingabe kommen.
Entsprechend sollte der Benutzer auf die Probleme hingewiesen werden.
In dem Modul Errorhandling werden Fehlermeldungen definiert und von anderen Modulen implementiert. Kapitel 2.3 gibt eine Übersicht über die verwendeten Fehlermeldungen.

\pagebreak

\subsection{Fehlerbehandlung}
Die Fehlerbehandlung wird im Modul Errorhandling organisiert.
In diesem wurden Kindklassen der Exception Klasse der Java Standardbibliothek implementiert.
Allgemein wurde versucht, möglichst wenig selbstdefinierte Exception Klassen zu erstellen.
In der Tabelle \ref{tab:exceptiontabel} sind die verwendeten Exceptions dargestellt.

\begin{table}[h]
	\centering
	
	\begin{tabularx}{\textwidth}{|X|X|}
		\hline
		\textbf{Exception} & \textbf{Quelle}  \\
		\hline
		\verb+IncorretInputFileException+ & Errorhandling-Modul  \\
		
		\verb+LexicalErrorException+ & Errorhandling-Modul  \\
		
		\verb+TypeMappingException+ & Errorhandling-Modul \\
		
		\verb+ParseException+ & Parser-Modul \\
		
		\verb+MappingException+ & Errorhandling-Modul \\
		
		\verb+IOExcpetion+ & Java-IO-Package \\
		
		\verb+NumberFormateExcpetion+ & Java-Lang-Package \\
		
		\verb+IllegalArgumentException+ & Java-Lang-Package \\
		
		\verb+NullPointerException+ & Java-Lang-Package \\
		
		\verb+DuplicateRequestException+ & Jdi-Request-Package \\
		\hline
		
	\end{tabularx}
	
	\caption{Liste der verwendeten Exceptions \label{tab:exceptiontabel}}
\end{table}
Die \verb+IncorretInputFileException+ wird geworfen, wenn
es sich bei der Eingabedatei nicht um eine PL/I-Datei handelt.
Der Transpiler akzeptiert lediglich das Dateiformat \verb+.pli+.

Enthält dabei ein zu übersetzender Ausdruck eine Variable des Typs Picture, soll die LexicalErrorException verhindern, dass ein fehlerhafter Picture-Ausdruck übersetzt wird.
Diese Fehlermeldung eignet sich weiterhin, um lexikalische von syntaktische Fehler zu unterscheiden. Da jedoch in der aktuellen Version des Transpilers die lexikalische und syntaktische Analyse von dem Parser-Modul übernommen werden, wird auf die ParseException zurückgegriffen.

Der Benutzer wird durch die \verb+ParseException+ darauf hingewiesen, wo ein Fehler auftritt und welche Schritte unternommen werden können, um den Fehler zu beheben.
Wird etwa ein syntaktisch falscher Ausdruck an das Parser-Modul übergeben, verarbeitet dieser den Ausdruck
bis zur Stelle, an der die definierte Grammatik nicht länger greift.
Der PL/I-Parser wirft die \verb+ParseException+ und gibt in dieser die Zeilen- und Spaltennummer des entstandenen Fehlers an.

Daraufhin überprüft das Checker-Modul nochmal die Zwischencodeerzeugung des Parsers.
Da das Checker-Modul lediglich einen Typ-Checker implementiert, existiert aktuell nur eine \verb+TypeMappingException+.
Diese wird immer dann geworfen, wenn der Typ einer deklarierten Variable mit einem unpassenden Wert zugewiesen wird.
Für weitere semantische Analyseschritte sind ebenfalls weitere Exceptions denkbar. Allgemein dient das Checker-Modul lediglich der Fehlererkennung und Unterbrechung des Transformationsprozesses.

Mithilfe der \verb+MappingException+ wird vor der Rückgabe der \verb+translate+ Methode des Mapper-Moduls überprüft, ob 
die benötigten Werte, korrekt aus dem Syntaxbaum verarbeitet wurden.
Es soll nach Möglichkeit vermieden werden, dass ein fehlerhaftes Programm transformiert wird. Die restlichen standart Java-Exceptions werden in dem zu erwartenden Zusammenhang eingesetzt.

In der aktuellen Version des Transpilers werden nicht alle PL/I-Syntaxelemente eingebunden.
In der PL/I-Sprachreferenz werden mehr als 300 unterschiedliche Funktionalitäten der Sprache beschrieben. \footcite[Vgl. ][S. 5ff.]{pliref}
Davon wurden ca. 65 in die Grammatik aufgenommen. Die Übersetzung dieser Strukturen ist aktuell durch 7 Unit-Tests abgedeckt.
Das hat zur Folge, dass es wahrscheinlich ist, dass der Benutzer auf eine ParseException stoßen wird.
Aufgrund der beschränkten Arbeitszeit wurden lediglich wesentliche Syntaxelemente der Programmiersprache implementiert.
Dazu zählen Deklarationen, Kontrollflussanweisungen, sowie Ein- und Ausgabeströme.
In dem nachfolgenden Kapitel werden Kernfunktionen vorgestellt und Übersetzungsentscheidung diskutiert. 

\pagebreak
\section{Technische Spezifikation}
\subsection{Ausführung des Transpilers}

% Importieren in Eclipse
Für die unterschiedlichen Benutzergruppen werden verschiedene Nutzungsmöglichkeiten des Transpilers angeboten. Diese umfassen die Verwendung über die \ac{ide} Eclipse und das Webinterface.

\paragraph*{Eclipse}
Die \ac{ide} Eclipse kann verwendet werden, um den Transpiler zu nutzen. Diese Herangehensweise ist besonders interessant für Administratoren, die den Quellcode des Transpilers selbst erweitern möchten.
Zuerst sollte das Projekt aus dem GitHub-Repository mit Git in das lokale Verzeichnis geladen werden.
Listing \ref{lst:gitclone} enthält den Git-Befehl zum Laden der benötigten Remote-Repositories. 

\begin{lstlisting}[language=bash, caption=Klonen der Transpiler Repositories, label={lst:gitclone}, basicstyle=\fontsize{9}{13}\selectfont\ttfamily]
 git clone -b integration 'https://github.com/lhahner/pl1-code-transpiler.git'; 
 git clone 'https://github.com/lhahner/plitra-web.git';
\end{lstlisting}

Daraufhin ist es möglich, das Maven-Projekt in Eclipse zu importieren. Unter der Registerkarte "Datei", kann das Projekt importiert werden. Im nächsten Menüfenster sollte der Ordner ausgewählt werden, in dem die \verb+pom.xml+ Datei liegt. Das Projekt wurde nun in Eclipse importiert. 

In dem Verzeichnis \verb+src/main/java/res/pli+ kann eine PL/I-Datei abgelegt werden, die übersetzt werden soll. Es ist auch möglich, das Standard-Eingabeverzeichnis zu ändern. Im Ordner \path{src/main/java/res/config} liegt die Datei \verb+config.properties+. In dieser Datei kann die Variable \verb+PATH+ geändert werden, um PL/I-Quellcode aus einer anderen Projektstruktur zu transformieren.

\paragraph*{Webanwendung}
Um es dem Benutzer leicht zu gestalten, den Transpiler zu verwenden, wurde mithilfe des Spring-Boot-Frameworks eine Webanwendung entwickelt.
Die Webanwendung enthält neben der Integration des Transpilers auch Dokumentationen der bisherigen Versionen.
Um die Anwendung zu starten, sollte mit der Kommandozeile in den Projektordner navigiert werden.
Es folgt das Starten des Spring-Boot-Projekts. Dazu ist in Listing \ref{lst:springboot} der zugehörige Befehl beschrieben.

\begin{lstlisting}[language=bash, caption=Build des Spring-Boot Projekts, label={lst:springboot}]
	.\mvnw spring-boot:run
\end{lstlisting}

Die Spring-Boot-Anwendung ist nun unter Port 8080 lokal erreichbar.
Hier kann entweder unter "Documentation" die Dokumentation der bisherigen Versionen eingesehen und unter "Transpile"
die eigentliche Anwendung aufgerufen werden.
Unter "Transpile" wird die grafische Oberflächliche aus Abbildung \ref{fig:translator} geöffnet.

\dhgefigure[h]{translate-view.png}[scale=0.48]{Grafische Oberfläche des Transpilers}{fig:translator}[][]

Auf der linken Seite der grafischen Oberfläche kann der PL/I-Quellcode eingefügt werden und auf der rechten Seite erscheint dann entweder der übersetzte Java-Zielcode oder eine Fehlermeldung.

Die Syntaxhervorhebung soll dem Entwickler ein vertrautes Gefühl geben.
Dazu wurde die Lösung Codemirror implementiert. Da Codemirror keine Unterstützung für PL/I bietet, wurde stattdessen die Syntax von PL/SQL verwendet. Die Java-Syntaxhervorhebung stammt ebenfalls von Codemirror. Wichtig ist an dieser Stelle anzumerken, dass die grafische Oberfläche derzeit nur lokal funktioniert und nach jeder Übersetzung neu gestartet werden muss. 
In den nachfolgenden Kapiteln werden PL/I-Strukturen beschrieben und die Transformation nach Java diskutiert.
\pagebreak

\subsection{Transformationsstrategien}
\subsubsection{Deklarationen und Zuweisungen}
Die erste Version des Transpilers enthielt lediglich die Transformation von PL/I-Datentypen in nicht-primitive Java Datentypen wie etwa die Klasse \verb+DECIMAL+, \verb+CHAR+ oder \verb+PICTURE+.

%In der Version des Transpilers aus der Projektarbeit IV wurden selbst geschriebene, nicht-primitive Datentypen verwendet, um die PL/I-Datentypen in entsprechenden Java-Zielcode zu übersetzen. Dies ist für Deklarationen ausreichend.

Bei der Verwendung dieser Datentypen in einer Programmroutine wurden jedoch Mängel des Designs sichtbar. Ein Beispiel hierfür ist die \verb+DECIMAL+-Klasse.

Werden in dem PL/I-Quellcode etwa die deklarierten Variablen verwendet, um mathematische Berechnungen zu erstellen, ist dies ohne Probleme mit den mathematischen Standardoperatoren wie \path{+} oder \verb+-+ möglich.

Um solche Ausdrücke auch in Java mit einem \verb+DECIMAL+-Objekt zu ermöglichen, muss die Klasse \verb+DECIMAL+ erweitert werden.
Denn in Java ist es nicht möglich, nicht-primitive Datentypen zu verrechnen, ohne vorher eine Methode zu schreiben, die diese Operation realisiert. 
Bei der Implementierung der Methoden können die Standart-Operatoren nicht überladen werden, wie es in C++ der Fall ist.

Listing \ref{lst:pliarithmeticexpression} demonstriert wie die Übersetzung eines arithmetischen Ausdrucks aussehen könnte.
Wäre in der Klasse \verb+DECIMAL+ für jede arithmetische Operation eine Methode definiert, wäre die Übersetzung wie in Listing \ref{lst:pliarithmeticexpression} denkbar.

%testing Side-by-side
\begin{minipage}[b]{0.48\linewidth}
	\centering
	\lstset{language=PL/I}
	\begin{lstlisting}[frame=single, numbers=left, mathescape,%
		caption={Transformation DECIMAL}, label={lst:pliarithmeticexpression}, basicstyle=\fontsize{9}{13}\selectfont\ttfamily]
 DCL var_1 FIXED DECIMAL(3) INIT(2);
 DCL var_2 FIXED DECIMAL(3) INIT(2);
 DCL var_3 FIXED DECIMAL(5);
		
 proc_1: PROC;
 	var_3 = var_1 + var_2;
 END proc_1;
	\end{lstlisting}
\end{minipage}
\hspace{0.5cm}
\begin{minipage}[b]{0.48\linewidth}
	\centering
	\lstset{language=Java}
	\begin{lstlisting}[frame=single, mathescape,%
		title={" "}, basicstyle=\fontsize{9}{13}\selectfont\ttfamily]
 DECIMAL var_1 = new DECIMAL(2).INIT(2);
 DECIMAL var_2 = new DECIMAL(2).INIT(2);
 DECIMAL var_3 = new DECIMAL(5);
		
 public void proc_1() {
 	var_3.INIT(var_1.add(var_2));
 }
	\end{lstlisting}
\end{minipage}

Für die weitere Übersetzung syntaktischer PL/I-Strukturen könnte die Entscheidung für eine solche Lösung die Lesbarkeit des übersetzten Programms beeinflussen. Beispielsweise könnte die Verwendung sowohl arithmetischer als auch boolescher Operatoren in einer Verzweigung den Java-Zielcode komplex wirken lassen. 

Wenn das \verb+DECIMAL+-Objekt durch einen primitiven Datentyp ersetzt werden würde, ist eine eins zu eins Transformation des zugewiesenen arithmetischen Ausdrucks in Listing \ref{lst:pliarithmeticexpression} möglich. Um die Längenbeschränkung aus PL/I auch in Java zu berücksichtigen, könnte die Implementierung einer Validierungsannotation erwogen werden.

Mithilfe der Jarkarta-Validations Software-Bibliothek kann eine Annotation erzeugt werden, die die Längenbeschränkung mittels Java-Reflection überprüft. \footcite[Vgl. ][]{jakarta}
Die Übersetzung des Ausdrucks ist in Listing \ref{lst:annotationdecimal} dargestellt.

\begin{minipage}[b]{0.48\linewidth}
	\centering
	\lstset{language=PL/I,label=SliceExaple}
	\begin{lstlisting}[frame=single, numbers=left, mathescape,%
		caption={Annotationslösung}, label={lst:annotationdecimal},
		basicstyle=\fontsize{9}{13}\selectfont\ttfamily]
 DCL var_1 FIXED DECIMAL(3) INIT(2);
 DCL var_2 FIXED DECIMAL(3) INIT(2);
 DCL var_3 FIXED DECIMAL(5);
		
 proc_1: PROC;
	var_3 = var_1 + var_2;
 END proc_1;
	\end{lstlisting}
\end{minipage}
\hspace{0.5cm}
\begin{minipage}[b]{0.48\linewidth}
	\centering
	\lstset{language=Java,label=SliceExaple}
	\begin{lstlisting}[frame=single, mathescape,%
		title={" "},
		basicstyle=\fontsize{9}{13}\selectfont\ttfamily]
 public @Decimal(3) double var_1 = 2;
 public @Decimal(3) double var_2 = 2;
 public @Decimal(5) double var_3;
		
 public void proc_1() {
 	var_3 = var_1 + var_2;
 }
	\end{lstlisting}
\end{minipage}

Durch diese Gestaltung ist die Einbindung und Erweiterung des übersetzten PL/I-Quellcode erleichtert. In diesem Zusammenhang wird die Längenbeschränkung mit Annotationen gelöst. 
So kann eine semantische nähe zu Java hergestellt werden und gleichzeitig der Wiedererkennungswert von PL/I beibehalten werden.

Da auch weitere PL/I-Datentypen wie etwa der \verb+PICTURE+ oder \verb+BINARY+ Typ mit dieser Lösung implementiert werden können, werden in zukünftigen Versionen des Transpilers auch hier die eigens definierten Klassen ersetzt.
Aktuell ist eine Zuweisung von Werten lediglich mit einer Variable des Typs \verb+CHAR+ und \verb+DECIMAL+ möglich.

Wie schon in Listing \ref{lst:annotationdecimal} und Listing \ref{lst:pliarithmeticexpression} zu sehen, werden in der aktuellen Version auch Prozeduren übersetzt.


\subsubsection{Programmstruktur und Programmablauf}
\paragraph*{Umwandlung von Programmstrukturen }\label{programstruct}

In PL/I werden Unterprogrammroutinen in Block-Strukturen definiert. Eine mögliche Blockstruktur ist die Prozedur. \footcite[Vgl. ][S. 97ff. ]{pliref}
Eine Prozedur wird durch den Bezeichner, das PL/I-Symbol \verb+PROCEDURE+ und eine Terminierung wie \verb+END+ beschrieben.
Zusätzlich können Parameter, Rückgabetypen und allgemeine Attribut Optionen definiert werden. In Listing \ref{lst:procchar} ist eine beispielhafte Prozedur gelistet.

\begin{minipage}[b]{0.48\linewidth}
	\centering
	\lstset{language=PL/I,label=SliceExaple}
	\begin{lstlisting}[frame=single, numbers=left, mathescape,%
		caption={Transformation einer Prozedur}, label={lst:procchar},
		basicstyle=\fontsize{9}{13}\selectfont\ttfamily]
 DCL Revenue FIXED DECIMAL(5);
		
 A10_Revenue: PROC(rev_1) RETURNS(DECIMAL(5)) OPTIONS(INLINE);
 	RETURN rev_1;
 END proc_1;
	\end{lstlisting}
\end{minipage}
\hspace{0.5cm}
\begin{minipage}[b]{0.48\linewidth}
	\centering
	\lstset{language=Java,label=SliceExaple}
	\begin{lstlisting}[frame=single, mathescape,%
		title={" "},
		basicstyle=\fontsize{9}{13}\selectfont\ttfamily]
 public @Decimal(5) double Revenue;		
		
 public @Decimal(5) double A10_Revenue(Object rev_1) { 
 	return (double)rev_1;
 }
	\end{lstlisting}
\end{minipage}

Die Prozedur \verb+A10_Revenue+ hat einen Parameter und gibt einen Wert des Typs \verb+DECIMAL+ der Länge 5 zurück.
Weiterhin wird die Option \verb+INLINE+ definiert.

Um in Java Unterprogramm-Routinen zu definieren, gibt es Methoden. 
Entsprechend sind Methoden ein mögliches äquivalent der Prozeduren. \footcite[Vgl. ][]{oracle}
Eine Methode hat ähnliche Bestandteile wie eine Prozedur.
Es wird ein Modifier, Rückgabetyp, Bezeichner und eine Parameterliste benötigt.
Einige dieser Attribute sind auch in einer Prozedur definiert.
Jedoch gibt es Unterschiede, die zu einer nicht eindeutigen Übersetzung führen können.

Wird etwa versucht, die in der PL/I-Prozedur definierten Parameter direkt zu übersetzen, fehlt die explizite Typisierung im Parameter. Dieser wird in PL/I implizit durch die Variablendeklaration definiert.
Es ist aber kein erforderlicher Bestandteil der Prozedurensignatur. 

Implizit wird hier der Typ \verb+DECIMAL+ zugewiesen. 
In Java muss hingegen explizit der Typ des Parameters in einer Methodensignatur angegeben werden.
Eine Möglichkeit, die Parameterliste zu übersetzen, ist über den Typ \verb+Object+. \footcite[Vgl. ][]{objectdocs}

In Zeile 4 des Listing \ref{lst:procchar} wird etwa in den Typ \verb+double+ gecastet. Weiterhin wird in der Prozedur aus Listing \ref{lst:procchar} ein Rückgabewert des Typs \verb+DECIMAL+ der L\"ange f\"unf definiert.

Auch hier eignet sich die Verwendung der in Kapitel 3.2.1 beschriebenen Annotationslösung. Dazu wird in Listing \ref{lst:procchar} über die Annotation die Längenbeschränkung definiert und der native Java-Typ \verb+double+ verwendet.
Die Validierung des Rückgabewertes erfolgt dann mithilfe von Boilerplate Code, der über die Codebasis bereitgestellt wird.

Ein weiteres Attribut, das in der Prozedur in Listing \ref{lst:procchar} definiert ist, ist das \verb+OPTIONS+
Attribut.
In PL/I werden hier Compiler-Optionen definiert. So wird, wie in Zeile 3 gezeigt, die Option \verb+INLINE+ verwendet, um den Kontext einer Prozedur zu beschreiben. Ist die Prozedur etwa \verb+INLINE+, ruft der Compiler nicht die Prozedur auf, sondern ersetzt den Aufruf mit dem Body der Prozedur. \footcite[Vgl.][]{optionsstmt} Diese sogenannte Inline-Expansion ist eine Optimierungsanweisung an den Compiler. Mit solchen Optionen kann der PL/I-Programmierer direkten Einfluss auf die Kompilierung des PL/I-Programms nehmen.

In Java wird hingegen der Java-Quellcode und häufig aufgerufene Methoden durch die Inline-Expansion des \ac{JIT} optimiert. Entsprechend existiert derselbe Optimierungsschritt auch in Java, kann jedoch nur bedingt beeinflusst werden. Die Deaktivierung dieser Optimierung würde zu einer Manipulation der Arbeitsweise des Java-Compilers führen. Es ist fraglich, ob eine solche Anpassung sinnvoll ist, da der Java-Compiler selbst entscheidet, welche Methoden durch die Inline-Expansion optimiert werden sollen. Ein Eingriff entspricht für \verb+javac+ im Allgemeinen keinem Optimierungsprozess.
Aus diesem Grund wird in der Version des Transpilers die Übersetzung dieses Ausdrucks nicht weiter berücksichtigt.

\paragraph*{Umwandlung des Programmablaufs}

In PL/I werden Prozeduren unter anderem mithilfe des \verb+CALL+ Statements aufgerufen. \footcite[Vgl. ][S.133ff. ]{pliref} Um einen Programmfluss zu erzeugen, wie er auch in einem nativen Java-Programm vorhanden ist, wird der Call-Ausdruck in einen Methoden-Aufruf transformiert. 
Das Beispiel in Listing \ref{lst:callstatement} beschriebt ein Programm, in dem sich die Prozeduren \verb+proc_1+ und \verb+proc_2+ gegenseitig aufrufen, was in der Main-Prozedur eine Endlosschleife simuliert.

\begin{minipage}[b]{0.48\linewidth}
	\centering
	\lstset{language=PL/I,label=SliceExaple}
	\begin{lstlisting}[frame=single, numbers=left, mathescape,%
		caption={Transformation Prozeduraufrufe}, label={lst:callstatement},
		basicstyle=\fontsize{9}{13}\selectfont\ttfamily]
 main_proc: PROC;
	 CALL proc_1;
 END main_proc;
		
 proc_1: PROC;
	 CALL proc_2;
 END proc_1;
		
 proc_2: PROC;
	 CALL proc_1;
 END proc_2;
	\end{lstlisting}
\end{minipage}
\hspace{0.5cm}
\begin{minipage}[b]{0.48\linewidth}
	\centering
	\lstset{language=Java,label=SliceExaple}
	\begin{lstlisting}[frame=single, mathescape,%
		title={" "},
		basicstyle=\fontsize{9}{13}\selectfont\ttfamily]
 public void main_proc(){
 	proc_1 ();
 }
		
 public void proc_1(){
	proc_2 ();
 }
		
 public void proc_2(){
	proc_1 ();
 }
	\end{lstlisting}
\end{minipage}


Die gleiche Endlosschleife aus PL/I wird somit auch in Java erzeugt.
Dadurch kann ein einfacher Programmablauf von einem PL/I-Programm in ein Java-Programm transformiert werden.
Um nun ein PL/I-Programm mit zusätzlicher Logik zu transformieren, sollten Verzweigungen und Schleifen ebenfalls übersetzt werden.

\pagebreak
\subsubsection{Kontrollflussanweisungen}
\paragraph*{Umwandlung von Boolesche Ausdrücke}
Für die Übersetzung von Kontrollflussanweisungen werden Boolesche Ausdrücke verwendet, um den Programmfluss basierend auf Wahrheitswerten zu verändern. Um sowohl Verzweigungen als auch Schleifen zu implementieren, müssen die Booleschen Ausdrücke von PL/I nach Java überführt werden. Da es in PL/I leicht unterschiedliche Boolesche Operatoren im Vergleich zu Java gibt, werden diese während der Übersetzung angepasst. In Tabelle \ref{tab:booloperator} sind die Booleschen Operatoren in PL/I sowie ihre Repräsentation in Java abgebildet.

% TODO Mappung Tabelle von boolesche Operationen, Tabelle: Boolesche-Operator -> PL/I-Symbol -> Java-Symbol

\begin{table}[h]
	\centering
	\begin{tabularx}{\textwidth}{|X|X|X|}
		\hline
		\textbf{Boolescher Operator} & \textbf{Symbol in PL/I} & \textbf{Symbol in Java}  \\
		\hline
		Gleich & = & ==  \\
		Und & \& & \&\& \\
		Nicht & ¬ & ! \\
		Oder & $\mid$ &	$\mid$$\mid$ \\
		Größer-als & $>$ & $>$ \\
		Kleiner-als & $<$ & $<$ \\
		Größer-Gleich & $>=$ & $>=$ \\
		Kleiner-Gleich & $<=$ & $<=$ \\
		\hline
		
	\end{tabularx}
	\caption{Boolesche Operatoren \label{tab:booloperator}}
\end{table}


% TODO (var_1 < 10) & ¬(var_2 > 5) übersetzen als Listing

Während in Java ein einfaches Ausrufezeichen eine Negation beschreibt, wird in PL/I das logische Negationszeichen (¬) verwendet. Bei der Transformation wird dieses Zeichen in ein Ausrufezeichen übersetzt.

Weiterhin wird in PL/I ein einzelnes Gleichheitszeichen als logischer Vergleichsoperator verwendet. Da in Java das einzelne Gleichheitszeichen der Zuweisung von Variablen dient, wird der PL/I-Vergleichsoperator mit dem Java-Vergleichsoperator ersetzt.

Gleiches gilt für das logische-Und, in PL/I wird lediglich ein Et-Zeichen verwendet. Während in Java zwei Gleichheitszeichen als logisches-Und verwendet werden.

In Listing \ref{lst:branchmapping} ist die Übersetzung eines Booleschen-Ausdrucks dargestellt.

\begin{minipage}[b]{0.48\linewidth}
	\centering
	\lstset{language=PL/I,label=SliceExaple}
	\begin{lstlisting}[frame=single, numbers=left, mathescape,%
		caption={Boolescher Ausdruck}, label={lst:branchmapping}]	
 (var_1 < 10) & $\lnot$(var_2 > 5)
	\end{lstlisting}
\end{minipage}
\hspace{0.5cm}
\begin{minipage}[b]{0.48\linewidth}
	\centering
	\lstset{language=Java,label=SliceExaple}
	\begin{lstlisting}[frame=single, mathescape,%
		title={" "}]
 (var_1 < 10) && !(var_2 > 5)
	\end{lstlisting}
\end{minipage}
\pagebreak

\paragraph*{Umwandlung von Verzweigungen}

In PL/I werden Verzweigungen mit den Symbolen \verb+IF+ und \verb+ELSE+ implementiert. 
Gepaart mit einem Booleschen Ausdruck, kann so ein Wahrheitswert abgefragt werden.
Somit lässt sich der Ausdruck in Listing \ref{lst:branchmapping} nach Java übersetzen.

\begin{minipage}[b]{0.48\linewidth}
	\centering
	\lstset{language=PL/I,label=SliceExaple}
	\begin{lstlisting}[frame=single, numbers=left, mathescape,%
		caption={Transformation Verzweigungen}, label={lst:branchmapping}]	
 IF (var_1 < var_2) THEN;
 	var_1 = 0;
 ELSE
 	var_1 = var_2;
 END;
	\end{lstlisting}
\end{minipage}
\hspace{0.5cm}
\begin{minipage}[b]{0.48\linewidth}
	\centering
	\lstset{language=Java,label=SliceExaple}
	\begin{lstlisting}[frame=single, mathescape,%
		title={" "}]
 if(var_1 < var_2){
 	var_1 = 0;	
 } else {
	var_1 = var_2;
 }
	\end{lstlisting}
\end{minipage} 

Würde hier erneut der nicht primitive Typ wie das \verb+DECIMAL+ Objekt verwendet, hätten die instanziierten Objekte erst mit einer Art \verb+toNumeric+ Methode in ein numerisches Format umgewandelt werden müssen. Was erneut zu mehr Quellcode als nötig führen würde. Alternativ könnte \verb+DEICMAL+ um eine \verb+compareTo(DECIMAL)+ Methode erweitert werden. In Java kann so das Interface \verb+Comparable+ implementiert werden. 

\paragraph*{Umwandlung von While-Schleifen}
In PL/I wird in der aktuellen Version des Transpilers, die  While- und Until-Schleife übersetzt.
Weitere Schleifen wie \verb+UPTHRU+ oder \verb+DOWNTHRU+ wurden in dieser Version nicht implementiert.
Da Java ebenfalls eine While-Schleife implementiert, kann diese als Übersetzungsmuster verwendet werden. 


Listing \ref{lst:whilecomamnd} zeigt eine einfache While-Schleife in PL/I.
Dabei wird in PL/I das Symbol \verb+DO+ nicht wie in Java für eine Do-While
Schleife verwendet, sondern leitet lediglich einen Schleifenausdruck ein.

\begin{minipage}[b]{0.48\linewidth}
	\centering
	\lstset{language=PL/I,label=SliceExaple}
	\begin{lstlisting}[frame=single, numbers=left, mathescape,%
		caption={Transformation While-Schleife}, label={lst:whilecomamnd}, basicstyle=\fontsize{9}{13}\selectfont\ttfamily]
 DCL var_1 DECIMAL(2) INIT(0);
 DCL var_2 DECIMAL(2) INIT(5);
		
 DO WHILE(var_1 < var_2);
 	var_1 = var_1 + 1;
 END;
	\end{lstlisting}
\end{minipage}
\hspace{0.5cm}
\begin{minipage}[b]{0.48\linewidth}
	\centering
	\lstset{language=Java,label=SliceExaple}
	\begin{lstlisting}[frame=single, mathescape,%
		title={" "}, basicstyle=\fontsize{9}{13}\selectfont\ttfamily]
 public @Decimal(2) double var_1 = 0;
 public @Decimal(2) double var_2 = 5;
		
 while(var_1 < var_2){
 	var_1 = var_1 + 1;
 }
	\end{lstlisting}
\end{minipage} 

Wie auch in dem PL/I-Programm Ausschnitt wird in dem übersetzten Java-Zielcode in Listing \ref{lst:whilecomamnd} eine Schleife erzeugt, die \verb+var_1+ solange erhöht, bis diese den Wert 5 erreicht hat.

\pagebreak
\paragraph*{Umwandlung von Until-Schleifen}

Die While-Schleife kann mit der Until-Schleife kombiniert werden. Eine Until-Schleife prüft eine Bedingung, deren Eintreten zum Abbruch der Schleife führt. Diese Form der Schleife existiert in Java nicht. Stattdessen wurde die do-while Schleife als Übersetzung gewählt. In Listing \ref{lst:untilwhile} ist die Übersetzung dargestellt.

\begin{minipage}[b]{0.48\linewidth}
	\centering
	\lstset{language=PL/I,label=SliceExaple}
	\begin{lstlisting}[frame=single, numbers=left, mathescape,%
		caption={Transformation Until-Schleife}, label={lst:untilwhile}, basicstyle=\fontsize{9}{13}\selectfont\ttfamily]
 DCL var_1 DECIMAL(2) INIT(0);
 DCL var_2 DECIMAL(2) INIT(5);
		
 DO WHILE(var_1 < var_2) 
		UNTIL(var_1 = 10);
 	var_1 = var_1 + 1;
 END;
		
	\end{lstlisting}
\end{minipage}
\hspace{0.5cm}
\begin{minipage}[b]{0.48\linewidth}
	\centering
	\lstset{language=Java,label=SliceExaple}
	\begin{lstlisting}[frame=single, mathescape,%
		title={" "}, basicstyle=\fontsize{9}{13}\selectfont\ttfamily]
 public @Decimal(2) double var_1 = 0;
 public @Decimal(2) double var_2 = 5;
		
 while (var_1 < var_2) {
  do {
	 var_1 = var_1 + 1;
	 } while(!(var_1 == 10));
 }
	\end{lstlisting}
\end{minipage} 

Um die Abbruchbedingung im Until-Teil der Schleife zu implementieren, wird die Boolesche Operation übersetzt und negiert. Dadurch kann die gleiche Laufzeit wie im PL/I-Programm erzeugt werden. Da der Do-Block mindestens einmal ausgeführt wird, jedoch nur einmal, wenn die Bedingung im Fuß der Do-While-Schleife erfüllt ist.

\subsubsection{Ein- und Ausgabe Befehle}
\paragraph*{Konsolen Ein- und Ausgabe}
Um eine Benutzerinteraktion zu ermöglichen, wird der Display-Ausdruck transformiert.
In PL/I gibt es sowohl die Möglichkeit, mit dem Display-Ausdruck eine Text-Nachricht in der Konsole auszugeben, sowie eine Benutzereingabe abzufragen. \footcite[Vgl. ][S. 264ff.]{pliref}

% TODO Kürzer
In Java gibt es verschiedene Implementierungen einer Umleitung der Ausgabe in die Konsole.
Die gängige Methode ist die Verwendung des Ausdrucks \path{System.out.println()}. 
Alternativ könnte auch die Methode \verb+log.trace+ verwendet werden.
Jedoch wird hier eine Abhängigkeit zu der Software-Bibliothek \verb+logger+  erzeugt, weshalb sich gegen diese Methode entschieden wurde.
Eine weitere Alternative würde der \verb+PrintWriter+ bieten, hierbei müsste jedoch zuerst das Objekt \verb+PrintWriter+ erzeugt werden, was zusätzlichen Quellcode erzeugen würde. Diese könnte mit der Implementierung des Sysout-Befehls vermieden werden.
Somit wurde mit dem Sysout-Befehl die gängige Methode gewählt. Entsprechend wird der Ausdruck in Listing \ref{lst:display} übersetzt.

\begin{minipage}[b]{0.48\linewidth}
	\centering
	\lstset{language=PL/I,label=SliceExaple}
	\begin{lstlisting}[frame=single, numbers=left, mathescape,%
		caption={Transformation Standardausgabe}, label={lst:display}, basicstyle=\fontsize{9}{13}\selectfont\ttfamily]
 DISPLAY ('Hello World');
	\end{lstlisting}
\end{minipage}
\hspace{0.5cm}
\begin{minipage}[b]{0.48\linewidth}
	\centering
	\lstset{language=Java,label=SliceExaple}
	\begin{lstlisting}[frame=single, mathescape,%
		title={" "}, basicstyle=\fontsize{9}{13}\selectfont\ttfamily]
 System.out.println("Hello World");
	\end{lstlisting}
\end{minipage} 


Um den Display-Ausdruck ebenfalls für das Einlesen von Benutzereingaben zu verwenden, wird an den bekannten Display-Ausdruck ein \verb+REPLY+ angefügt. Siehe Listing \ref{lst:displayreply}.

\begin{minipage}[b]{0.48\linewidth}
	\centering
	\lstset{language=PL/I,label=SliceExaple}
	\begin{lstlisting}[frame=single, numbers=left, mathescape,%
		caption={Transformation Standardeingabe}, label={lst:displayreply}, basicstyle=\fontsize{9}{13}\selectfont\ttfamily]
 DCL username CHAR(5);
 DISPLAY('Username') REPLY(username)
	\end{lstlisting}
\end{minipage}
\hspace{0.5cm}
\begin{minipage}[b]{0.48\linewidth}
	\centering
	\lstset{language=Java,label=SliceExaple}
	\begin{lstlisting}[frame=single, mathescape,% 
		title={" "}, basicstyle=\fontsize{9}{13}\selectfont\ttfamily]
 public @Char(5) String username;
 username = System.console().readLine(); 
	\end{lstlisting}
\end{minipage} 

In dem Beispiel in Listing \ref{lst:displayreply} wird die Benutzereingabe in der
Variable \verb+username+ gespeichert.
In PL/I kann die Benutzereingabe nur in einem Bit, Widechar oder Char gespeichert werden. 
In Java gibt es die Möglichkeit über den Scanner oder BufferedReader eine Benutzereingabe abzufragen.

Hier wird das Objekt System verwendet, welches auch bei der \verb+System.out.println+ verwendet wurde.
Mit \verb+System.console.readLine()+ wird die Benutzereingabe gelesen.

\paragraph*{Datei Eingabe}
In PL/I gibt es die Möglichkeit, über das \verb+READ+-Statement Dateien vom Dateisystem einzulesen.
In Kapitel 1.1 wurde bereits erwähnt, dass PL/I-Programme hauptsächlich auf einem Mainframe ausgeführt werden.
Auf einem z/Os-Betriebssystem wird das Dateisystem \ac{zfs} verwendet.
Hier werden, anders als bei den  Linux oder Windows \ac{hfs}s, Ordner als Datasets und Dateien als Records gespeichert.
Ein Record kann dann von einem Programm, wie einem PL/I-Programm, gelesen werden.

Ein PL/I-Programm kann einen Record verarbeiten sowie ausgeben. In Java geschieht dies gewöhnlich über die Angabe eines Dateipfads.
In PL/I werden spezifische Dateien über die Batch-Verarbeitung definiert, weshalb im Quellcode selbst keine Referenz zu dem eigentlichen Record gegeben ist.

Um den durch die Batchverarbeitung definierten Record in einem PL/I-Programm zu verwenden, kann der Datentyp \verb+FILE+ in Listing \ref{lst:plifiletyp} definiert werden. Zusätzlich können Attribute angegeben werden, wie \verb+BUFFERED+ oder \verb+UNBUFFERED+ um dem Compiler exakt mitzuteilen, wie dieser die Datei verarbeiten soll.

\begin{minipage}[b]{0.48\linewidth}
	\centering
	\lstset{language=PL/I,label=SliceExaple}
	\begin{lstlisting}[frame=single, numbers=left, mathescape,%
		caption={Transformation Dateityp}, label={lst:plifiletyp}]
 DCL file_1 FILE;
	\end{lstlisting}
\end{minipage}
\hspace{0.5cm}
\begin{minipage}[b]{0.48\linewidth}
	\centering
	\lstset{language=Java,label=SliceExaple}
	\begin{lstlisting}[frame=single, mathescape,%
		title={" "}]
 public File file_1 = new File("");
	\end{lstlisting}
\end{minipage}  

Bei dem Typ FILE handelt es sich um einen Buffer, in dem der Inhalt einer Datei gelesen und für weitere Programmroutinen verfügbar wird. \footcite[Vgl. ][ S.305ff. ]{pliref}

In Java existiert ein ähnlicher, nicht-primitiver File-Typ.
Java liest mit der Klasse File eine Datei direkt vom Betriebssystem ein und speichert den Inhalt in einem Buffer. 
Hingegen besteht in Java kaum die Möglichkeit, dabei direkt zu beeinflussen, wie die Datei eingelesen werden soll.


Bezüglich Listing \ref{lst:plifiletyp} ist anzumerken, dass bisher lediglich ein File-Objekt instanziiert wurde.
Es wurde kein Pfad übergeben und somit auch keine Datei eingelesen. 
Die Ausführung dieses Java-Codes wird zu einer Input-Output-Exception führen, da kein Pfad zu einer Datei übergeben wurde. Hier ist die eigentliche Übersetzung des PL/I-Programms abhängig von der Betriebssystemumgebung, in der das Java-Programm ausgeführt werden soll. Für den Transpiler ist ohne die Referenz auf die Batch-Verarbeitung eine Einbindung des Dateipfades nicht möglich. Entsprechend wird in dieser Version des Transpilers dem Benutzer die Verantwortung übergeben, die Referenz zu den ursprünglichen Eingabedateien herzustellen. In PL/I wird mit einem \verb+READ+-Ausdruck die definierte Datei gelesen und in einer alphanumerischen Variable gespeichert. Ein mögliches \verb+READ+-Statement aus einem PL/I-Programm ist in Listing \ref{lst:pliread} dargestellt. Hierbei wird die in Listing \ref{lst:pliread} definierte Datei gelesen. Transformiert wird diese Dateieingabe mit dem Scannerobjekt.

\begin{minipage}[b]{0.48\linewidth}
	\centering
	\lstset{language=PL/I,label=SliceExaple}
	\begin{lstlisting}[frame=single, numbers=left, mathescape,%
		caption={Transformation Dateieingabe}, label={lst:pliread}, basicstyle=\fontsize{9}{13}\selectfont\ttfamily]
 DCL var_1 CHAR(4);
 READ FILE(file_1) INTO (var_1); 
		
		
		
		
		
	\end{lstlisting}
\end{minipage}
\hspace{0.5cm}
\begin{minipage}[b]{0.48\linewidth}
	\centering
	\lstset{language=Java,label=SliceExaple}
	\begin{lstlisting}[frame=single, mathescape,%
		title={" "},  basicstyle=\fontsize{9}{13}\selectfont\ttfamily]
 String @Char(4) var_1;
 Scanner readFile = new Scanner(file_1);
		
 while(readFile.hasNext()){
	var_1 = readFile.next();
 }
 readFile.close();
	\end{lstlisting}
\end{minipage}  


Mit dem Scanner Objekt kann die definierte Datei nun Wort für Wort in einen String gelesen werden.

\paragraph*{Datei Ausgabe}

Neben dem Einlesen kann ein PL/I-Programm auch, mit dem \verb+WRITE+ Befehl, in eine Datei schreiben. Hierbei ist zu erwähnen, dass die eigentliche Steuerung der Ausgabe durch das Batch-Programm vorgenommen wird.
Es wird der Buffer der vorher definierten Variable des Typs \verb+FILE+ benutzt, um in die Datei bzw. den Record
des Datasets zu schreiben. Dabei wird wie in diesem Fall der Inhalt des Strings \verb+var_1+ gelesen und entsprechend
in den Buffer geschrieben.
Die Übersetzung des PL/I-Quellcode führt zu der Ausgabe in Listing \ref{lst:javawriter}.

\begin{minipage}[b]{0.48\linewidth}
	\centering
	\lstset{language=PL/I,label=SliceExaple}
	\begin{lstlisting}[frame=single, numbers=left, mathescape,%
		caption={Transformation Dateiausgabe}, label={lst:javawriter},  basicstyle=\fontsize{9}{13}\selectfont\ttfamily]
	WRITE FILE (file_1) FROM (var_1);	
	
	
	
	
	\end{lstlisting}
\end{minipage}
\hspace{0.5cm}
\begin{minipage}[b]{0.48\linewidth}
	\centering
	\lstset{language=Java,label=SliceExaple}
	\begin{lstlisting}[frame=single, mathescape,%
		title={" "},  basicstyle=\fontsize{9}{13}\selectfont\ttfamily]
	BufferedWriter writer = new BufferedWriter(
		new FileWriter(file_1)
	); 
	writer.write(var_1);
	\end{lstlisting}
\end{minipage}  

Hierbei wird ein \verb+BufferedWriter+-Objekt erzeugt, dem ein \verb+FileWriter+ übergeben wird. Dies ist eine mögliche Methode, um in Java eine Datei mit Inhalt zu füllen. Ähnlich wie bei der Standardeingabe können auch weitere Objekte aus ausgewählten Java-Bibliotheken verwendet werden, wie etwa der \verb+FileWriter+, \verb+PrintWriter+ oder beide in Kombination mit dem \verb+DataOutputStream+-Objekt. Bei den letztgenannten Lösungen werden mehr als zwei Objekte für das Schreiben in die Datei benötigt.
Zusammenfassend werden so die grundlegenden Anweisungen der Programmiersprache PL/I nach Java übersetzt. 

\pagebreak
\subsection{Test und Integration}
\subsubsection{Unit-Tests}

Um die in Kapitel 3.2 beschriebenen Übersetzungen zu testen, wurden Unit-Tests geschrieben. 
Diese Unit-Tests testen auf der Basis bisher vorgestellter Übersetzungen den Transpiler. Die Vorgehensweise bei der Entwicklung des Transpilers orientiert sich am Test-Diven-Development. 
Der erste Schritt bestand in der Konzeption von Elementarübersetzungen wie in der technischen Spezifikation vorgestellt.
Für jede dieser Elementarübersetzungen wurde ein Unit-Test geschrieben. Diese werden mit dem Maven-Kommando \verb+mvn test+ als Teil des Buildprozesses durchgeführt. Daraufhin folgt die Implementation der Programmroutinen, um den PL/I-Quellcode nach Java zu übersetzen.
Beispielhaft ist in Listing \ref{lst:javaunittest} eine Test-Methode für das Transformieren von Variablen dargestellt.

\begin{lstlisting}[language=Java, caption=Positvtest für die Übersetzung eines arithmetischen Ausdrucks, label={lst:javaunittest}]
@Test
void mapChildNodes_checkIdentifierMapping() {
	
	DeclarationMapper declarationMapper = new DeclarationMapper();
	String decimalExpression = 
		 "test_1_package: PACKAGE;" 
	   + "	DCL var_3 FIXED DECIMAL(5);" 
	   + "END test_1_package;";
	
	InputStream stream = new ByteArrayInputStream
	(decimalExpression
	.getBytes(StandardCharsets.UTF_8));
	
	Pl1Parser pl1parser = new Pl1Parser(stream);
	SimpleNode varNode = pl1parser.program().jjtGetChild(0)
	.jjtGetChild(1);
		
	declarationMapper.mapChildNodes(varNode);
	declarationMapper.mapArithmetic
	((SimpleNode)varNode
	.jjtGetChild(1).jjtGetChild(0));
		
	assertEquals("var_3", declarationMapper.getIdentifier());
	assertEquals("@Decimal(5) double", declarationMapper.getType());
}
\end{lstlisting} 

% Test-Coverage
In Listing \ref{lst:javaunittest}  wird der Typ \verb+DECIMAL+ getestet. Untersucht werden der resultierende Typ und Bezeichner, die durch die Methoden \verb+mapChildNodes+ und \verb+mapArithmetic+ gesetzt werden. Es wird erwartet, dass der Bezeichner der Variable aus PL/I dem Inhalt der Variable identifier der Klasse \verb+DeclarationMapper+ entspricht.
Weiterhin wird erwartet, dass der Typ \verb+DECIMAL+ zum entsprechenden Typ aus Listing \ref{lst:annotationdecimal} übersetzt wird.
Mit jedem Test wird auch ein Parser-Objekt erzeugt, dadurch wird  mit einem Unit-Test die geschriebene Grammatik überprüft.
Ähnlich wurden  für die weiteren Mapper-Klassen Test-Klassen erzeugt.
Für die restlichen Module, wie etwa der Symboltabelle, wurden ebenfalls Tests nachträglich geschrieben.

%Negative Tests
Bei der Implementierung der Tests wurden sowohl Negativ- als auch Positivtests integriert.
In Listing \ref{lst:javaunittest} handelt es sich um einen Positivtest. Da die Methode \verb+mapArithmetic+ speziell für die Übersetzung von numerischen Typen geeignet ist, sollte die Methode eine Fehlermeldung ausgeben, wenn eine Übersetzung von einem alphanumerischen Datentyp verlangt wird.  In Listing \ref{lst:javanegativetest} wird dieser Umstand getestet. 

\begin{lstlisting}[language=Java, caption=Negativtest für die Übersetzung eines arithmetischen Ausdrucks, label={lst:javanegativetest}]
@Test
void mapArithmetic_NegativeTest() {
	DeclarationMapper declarationMapper = new DeclarationMapper();
	
	String decimalExpression = 
	"test_1_package: PACKAGE;" 
	+ "	DCL var_3 CHAR(5)" 
	+ "END test_1_package;";
	
	InputStream stream
	= new ByteArrayInputStream(decimalExpression
	.getBytes(StandardCharsets.UTF_8));
		
	Pl1Parser pl1parser = new Pl1Parser(stream);
	SimpleNode program = pl1parser.program();
	SimpleNode varNode = 
	(SimpleNode) program
	.jjtGetChild(0)
	.jjtGetChild(1);
		
	assertThrows(TypeMappingException.class, () -> 
	{
		declarationMapper.mapArithmetic(
		(SimpleNode)varNode.jjtGetChild(1).jjtGetChild(0)
		);
	});
}
\end{lstlisting}

Weiterhin wurden Negativ- und Positivtests für alle weiteren Mapper-Klassen geschrieben. Diese finden sich in Anhang A.
Zusammenfassend wurden so die kleinsten Programm-Einheiten des Transpilers überprüft. 


\subsubsection{Integrationstests}
In diesem Kapitel wird die Durchführung eines Integrations-Tests aller Komponenten beschrieben. Der Integrations-Test ist so angelegt, dass im ersten Schritt der Transpiler kompiliert und danach ein Beispielprogramm übersetzt wird. Der übersetze Java-Code wird wiederum kompiliert und ein Test des Programms in Kapitel 3.4 durchgeführt.

In diesem Kapitel wird die vollumfängliche Implementation des Transpilers, mit dem Frontend und der zum Kompilieren benötigte Codebasis protokolliert. Dazu wird ein Beispiel PL/I-Programm übersetzt, welches den Probedivisions Algorithmus zum berechnen von Primzahlen implementiert. 
In Anhang \ref{lst:pliprobedivision} ist die Implementation des Algorithmus in PL/I zu sehen. Dieses Programm soll während des Integrations-Tests in Java-Zielcode übersetzt und durch den Java-Compiler kompiliert werden.
Eine Ausführung des Algorithmus ist nicht vorgesehen, lediglich die Transformation und das Kompilieren durch den PL/I-Transpiler sowie des Java-Compilers sind vorgesehen. Das System verwendet die Version 17.0.11 des Java-Compilers. Sowie Version 5.0 des Parser-Generators-JavaCC und Apache Maven der Version 3.6.3. Der Integrationstest beginnt damit, das Transpiler-Projekt aus dem Remote-Repository zu laden.

\begin{lstlisting}[language=Bash, caption=Klonen des Transpilers, label={lst:inttest1}]
git clone -b integration https://github.com/lhahner/pl1-code-transpiler.git
\end{lstlisting} 

Es wird in das Projektverzeichnis des Transpilers navigiert, woraufhin die JAR-Datei mit dem Maven-Kommando in \ref{lst:inttest2} Listing erzeugt wird.

% Transpiler kompilieren
\begin{lstlisting}[language=Bash, caption=Erzeugen der JAR, label={lst:inttest2}]
mvn install
\end{lstlisting}

% Frontend aus Github repo holen
Jetzt kann das Projekt in das Frontend eingebunden werden. Dazu wird erneut das Frontend-Projekt aus dem Remote-Repository geladen. In Listing \ref{lst:inttest3} ist der Befehl dazu gelistet.

\begin{lstlisting}[language=Bash, caption=Klonen des Frontend-Projekts, label={lst:inttest3}]
git clone https://github.com/lhahner/plitra-web.git
\end{lstlisting}

% Server stareten
Nun wird in die Projektstruktur des Frontendprojektes navigiert. Der Spring-Boot-Server wird lokal mit dem Befehl in Listing \ref{lst:inttest4} gestartet.

\begin{lstlisting}[language=Bash, caption=Starten des Spring-Boot-Servers, label={lst:inttest4}]
mvn spring-boot:run
\end{lstlisting}

Die Adresse \path{http://localhost:8080/} wird in Firefox geöffnet. Es wird über den \emph{Transpile} Button auf die Seite \path{http://localhost:8080/trans} navigiert. Hier wird in das linke Textfeld der PL/I-Quellcode aus Anhang \ref{lst:pliprobedivision} eingefügt. In Anhang \ref{lst:javaprobedivision} ist das Ergebnis, welches auf der rechten Seite ausgegeben wurde, dargestellt.

% Code base Repository aus Github holen
Um nun das ausgegebene Projekt zu kompilieren, sollte eine bereitgestellte Codebasis verwendet werden. Es wird das Code-Basis-Projekt aus der Remote-Repository in Listing \ref{lst:inttest5} geladen.

\begin{lstlisting}[language=Bash, caption=Klonen der Codebasis, label={lst:inttest5}]
git clone https://github.com/lhahner/plitra-codebase.git
\end{lstlisting}

% Code integration des Übersetzen Codes in Repositry
Der Code innerhalb der Klasse aus Anhang \ref{lst:javaprobedivision} wurde kopiert und in die Klasse TranspiledExamplePliProgram eingefügt. 

Dazu wurde in den Ordner \path{/src/main/java/org/plitra/codebase} innerhalb des Codebasis-Projektes navigiert und mit Eclipse der Java-Zielcode eingefügt.

% kompilieren des Übersetzen Codes
Das Casten des Parameters wird aktuell nicht unterstützt, weshalb der Objekttyp händisch
in \verb+double+ geändert wurde. Weiterhin wurden alle Fehler, die von der \ac{ide} erkannt wurden, automatisch durch das Akzeptieren, der Änderungsvorschlage der  \ac{ide}  verbessert.
Das Projekt wird nun erneut mithilfe des in Listing \ref{lst:inttest6} dargestellten Maven-Kommandos kompiliert.

\begin{lstlisting}[language=Bash, caption=Kompilieren des Projekts, label={lst:inttest6}]
mvn install -Dmaven.test.skip=true
\end{lstlisting}

Das Projekt wurde erfolgreich kompiliert, womit der Integrations-Test erfolgreich war. Um nun weitergehend auch die Laufzeit des Java-Zielcodes zu überprüfen, wird in dem nächsten Kapitel die Performance des PL/I-Quellcodes und des Java-Zielcodes überprüft und verglichen. 
\pagebreak

% TODO Code in Textdatei speichern -> um Main-Methode anreichern -> mit Javac kompilieren -> Performance und Benchmark ausführen können

\subsection{Performance und Benchmarks}
Um vergleichen zu können, wie sich die Laufzeit der Implementierung des Probedivisionsalgorithmus in PL/I und Java unterscheidet, wird in diesem Kapitel die Performance verglichen. 
Dazu werden den beiden Programmen vier Primzahlen übergeben. Für die Berechnung wurde das Programm aus Anhang \ref{lst:pliprobedivision}, auf einem z/Os-Mainframe-System kompiliert und ausgeführt. 
Die allgemeinen System-Spezifikationen des Mainframes sind in Abbildung \ref{fig:mainframesysteminfo} dargestellt.

\dhgefigure[h]{mainframe-system-info.png}[scale=0.46]{Mainframe System Spezifikation}{fig:mainframesysteminfo}[][]

Unter CPU in Abbildung \ref{fig:mainframesysteminfo} ist der Maschinen-Typ 3931 gelistet. Bei dem Mainframe handelt es sich um einen IBM z16 A02 mit
maximal 16 Prozessorkernen. Sowie maximal 4 TB Arbeitsspeicher. \footcite{z16}

Damit wird das Programm auf der aktuell (Juli 2024) neusten Hardware von IBM kompiliert und ausgeführt. Zum Vergleich wird das während des Integrationstest übersetzte Programm aus Anhang \ref{lst:javaprobedivision} ebenfalls für die Berechnung der Primzahlen getestet. 
Die Hardware-Spezifikationen des Systems, auf dem der Integrationstest durchgeführt wird, sind in Tabelle \ref{tab:hardwartable} beschrieben.

\begin{table}[h]
	\centering
	\begin{tabularx}{\textwidth}{|X|X|X|}
		\hline
		\textbf{Spezfikation} & \textbf{Beschreibung}  \\
		\hline
		Betriebssystem & Ubuntu 20.04.6 LTS x86 64 \\
		
		Kernel & 5.15.0-113-generic  \\
		
		CPU & Intel i3-9100F (4) @ 3.600GHz \\
		
		GPU & NVIDIA GeForce GTX 970  \\
		
		Arbeitsspeicher & 8217MiB / 15932MiB \\
		\hline
		
	\end{tabularx}
	\caption{Hardwarespezifikation des Testsystems \label{tab:hardwartable}}
\end{table}


% PL/I-Benchmark Test mit System Spezifikationen
Um ein PL/I-Programm auf einem z/Os-Betriebssystem eines IBM-Mainframes zu starten, muss das Programm zuerst erfolgreich kompiliert und mit einem Batchjob-Programm ausgeführt werden. Das Batchjob-Programm wurde in der Job-Control-Language (JCL) und nach jedem Testlauf eine größere Primzahlen übergeben.
Die Grundlage des Benchmarktests sind die bisher größten bekannten Primzahlen der Jahre 1772-1876.\footcite[Vgl. ][]{prime} In Tabelle \ref{tab:plibenchmark} sind alle getesten Primzahlen mit ihren Laufzeiten zu sehen. 

\begin{table}[h]
	\centering
	\begin{tabularx}{\textwidth}{|X|X|X|}
		\hline
		\textbf{Primzahl} & \textbf{PL/I Laufzeit} & \textbf{Java Laufzeit} \\
		\hline
		2,147,483,647 & 00:00:00,00 & 00:00:00,21 \\
		
		999,999,000,001 & 00:00:00,01 & 00:00:01,20\\
		
		67,280,421,310,721 & 00:00:00,08 & 00:00:05,90 \\
		
		10,000,000,002,065,383 & 00:00:01,00 & 00:01:27,90 \\
		\hline
	\end{tabularx}
	\caption{PL/I-Programm Benchmarks (siehe Anhang C.1 - C.4) \label{tab:plibenchmark}}
\end{table}


Die Laufzeiten aus Tabelle \ref{tab:plibenchmark} zeigen, dass der Mainframe für die Berechnung der Primzahlen maximal eine Sekunde braucht. Für die Berechnung der Primzahlen benötigte das Javaprogramm für fast jede mehr als eine Sekunde. Bei der Beurteilung sollten die Hardwareunterschiede zwischen dem IBM-Mainframe und dem Linux-Computer berücksichtigt werden. Es ist zu beobachten das die Laufzeit stabil um den Faktor 100 größer ist im Vergleich zum Mainframe.

Eigentlich wäre anstelle der Zahl 10,000,000,002,065,383 die Primzahl 170,141,183,460,469,- \newline 231,731,687,303,715,884,105,727 in der Liste der größten bekannten Primzahlen an der Reihe gewesen. 
Da diese Zahl die maximale Anzahl an Ziffern übersteigt, die ein primitiver Datentyp in Java verarbeiten kann, wurde stattdessen der nicht-primitive Datentyp \verb+BigDecimal+ verwendet. Im Anhang \ref{lst:javaprobedivisionang} befindet sich das angepasste Programm. 
Dies wurde im Benchmark nicht berücksichtigt da die Berechnung mit dem System aus Tabelle \ref{tab:hardwartable} nach 9h abgebrochen wurde und sich dafür entschieden wurde auch diese Zahl aus Kostengründen auf dem Mainframe nicht zu testen. Auf dem Mainframe werden Batchprogramme nach \ac{MIPS} abgerechnet.

Zusammenfassend wird so eine Schwäche des Transpilers deutlich. Durch die Überschreitung von Speichergrenzen sollte der Transpiler entsprechend eine geeignetere Typisierung wählen. In der aktuellen Version ist die differenzierte Typisierung je nach Wertzuweisung noch nicht implementiert. Der Administrator müsste in diesem Fall nur das Mapper-Modul des Transpilers anpassen. In dem Mapper-Modul muss das Template Enum entsprechend der Operationen des BigDecimals angepasst werden, sodass im BooleaExpressionMapper der Vergleichsoperator mit der \verb+compareTo+-Methode übersetzt werden.
\pagebreak

\section{Fazit}
% Erreichte Entwicklung
Der entwickelte Transpiler bietet die Voraussetzungen für die Übersetzung von PL/I nach Java. Dazu wurde eine PL/I-Grammatik geschrieben, die es ermöglicht, mit JavaCC einfache Programme der Sprache PL/I zu verarbeiten. Während dieser Arbeit wurden die aufgesetzten Ziele im wesentlichen erreicht. 
Einerseits wurde der Transpiler in einer modularen Architektur entwickelt. Dabei haben sich mehrere Entwurfsmuster als nützlich erwiesen,
namentlich die Verwendung der Strategie-, Singleton- und Composite- Entwurfsmuster.
Es wurde in den Buildprozess auch automatisiertes Testen eingebunden, um die Korrektheit von Elementarübersetzungen sicherzustellen. Das gewährleistet die hier erreichte Qualität auch bei der zukünftigen Weiterentwicklung.
Darüber hinaus ist es nun auch Benutzern möglich, den Transpiler über eine grafischen Oberfläche zu verwenden.
Mit der Entscheidung für ein Webfrontend wurde die Anforderung der Plattformunabhängigkeit erreicht. Damit lässt sich die Reichweite des Transpilers und damit der Programmiersprache PL/I insgesamt erweitern.
Der Performancetest hat ergeben, dass durch den Wechsel auf leistungsschwächere Hardware bei der Laufzeit rechenintensiver Anwendungen Einbußen um den Faktor 100 realistisch sind.

Viele Konzepte aus der Projektarbeit IV mussten verbessert und angepasst werden, was dazu führte, dass nahezu 90\% des Transpilers
während dieser Arbeit abgeändert wurden. Dies hatte allgemein positive Auswirkungen, da die Entwicklung nun wesentlich leichter fortgeführt werden kann.

% Ausblick, Schwächen
\paragraph*{Ausblick} Die Übersetzung von Datentypen wurden erweitert und mit der Implementierung von Java-Reflection und Annotationen wurde eine lesbare und fast native Java-Übersetzung gefunden. Leider ist noch nicht für jeden Datentyp eine solche Annotation implementiert. 

Der Benutzer wird schnell einen Ausdruck finden, der für einen vollumfänglichen Transpiler zu erwarten wäre, jedoch in diesem nicht implementiert wurde.
Auch wird der Benutzer feststellen, dass der PL/I-Quellcode nicht vollumfänglich auf Fehler überprüft wird. Eine Mitübersetzung von Fehlern ist somit denkbar.
Die Grammatik deckt aktuell weniger als 20\% des Sprachumfangs ab. Unit-Tests decken davon wiederum 10\% ab.
Diese wird verbessert, sodass die PL/I-Sprachreferenz nahezu vollständig eingearbeitet werden kann.

Es ist dabei auch weiterhin fraglich, ob das übersetzte Java-Programm den Zweck des PL/I-Programms vollständig erfüllt. 
Um dies sicherzustellen, sollten neben Unit-Tests, Integrations-Tests und System-Tests auch User-Acceptence-Tests berücksichtigt werden.

Aktuell ist mit einer Singleton-Symboltabelle keine Verarbeitung paralleler Requests möglich. Mit Erweiterung des Singleton- zum Factory-Entwurfsmuster wäre es möglich, verschiedene PL/I Dateien parallel zu übersetzen und im Webfrontend mehrere Nutzeranfragen parallel zu verarbeiten. 

Für den PL/I-Entwickler wurden die ersichtlichen Strukturen eines PL/I-Programms transformiert. Weitergehend werden Übersetzungen der JCL-Steuerkarten benötigt, 
um den Kontext eines PL/I-Programms vollständig zu erfassen. Als Erweiterung der bisherigen Arbeit wäre damit eine Integration in das Spring-Batch-Framework denkbar.


% Wofür sind die PL/I Batchprogramme zuständig

% Wie werden die Batchprogramm gesteuert (JCL - Organisation verstehen)

% Wie bildet du JCL in Java ab -> Die Antwort lautet Spring Batch Framework ist der ersatz von JCL im Java-Kontext aufm Jarkarta Stack -> wie kannst du die JCL Kontext Parameter in eine geeignete Spring Batch verarbeitung/Konfiguration integrieren


